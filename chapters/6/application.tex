\chapter{Application}
\label{chapter:application}

{\section{Framework Application}
To illustrate how the framework can be applied to evaluate current Web applications and suggest new tooling, we use it to examine one of the cases of the study, Google Maps. By answering the questions from the framework, we get the following description of Google Maps.

\textbf{Serendipitous discovery.} Although there are some possibilities for serendipitous discovery within Google Maps, it is limited by a few factors. Arbitrary navigation is only possible when the user looks at the map itself or browses through the images of nearby places. Any other information discovery must be initiated by search, and thus, the user needs to formulate their information need---the application lacks category-based navigation, so there is nothing that aids users in the formulation of an information need. Once the application returns search results, the possibility for serendipitous information discovery increases. Some interesting information can be discovered on the business' official Website or integrated Google+ page that the user can navigate to by clicking on 'reviews'. However, the 'reviews' link doesn't have a visual preview to indicate that there are more than just reviews on the linked page. Considering the nature of Google Maps, the semantic of the spatial arrangement of resources is defined by the locations of actual places on the map. More information is presented as a list. 

\textbf{Fact finding.} Fact finding is well supported in Google Maps. Since the application provides access to only one type of resource (places), there is no need for category-based navigation. Direct navigation is not always possible, but some places are visible on the map so the user can click on a place and the application will display relevant information. Search-based navigation within Google Maps is usually precise and returns accurate search results for specific places. The application is conveniently integrated with Google+, allowing access to relevant information, such as reviews, images, and hours of operation. Resources are displaced in a uniform fashion making it easy to find information such as addresses and contacts. 

\textbf{Rediscovery.} There are a few ways to rediscover information through Google Maps. Google Maps employs a history-based revisitation mechanism, so users can see the last few places they searched for when opening the page. Users can bookmark a place on a list called "My Places" by clicking on the 'star' icon. Lastly, it is easy to rediscover information about a place by simply searching for it. Returned results are usually both accurate and reliable.

\textbf{Channel-based discovery.} Channel-based rediscovery is common among applications with content that is frequently updated. Content provided by Google Maps is fairly stable, and therefore, there are no channel-based discovery mechanisms used by the application.

\textbf{Management.} Google Maps does not allow the creation of custom lists nor does it allow tagging. Users can only bookmark places to the "My Places" list. 

\textbf{Preservation.} Personal preservation in Google Maps is possible through adding the place to the "My Places" list as mentioned above---by adding internal content to internal storage. Other types of place preservation are possible through Google+, however, not within Google Maps.

\textbf{Augmentation.} Users can evaluate and annotate places through Google+. However, aggregated reviews and ratings are visible on Google Maps. 

\textbf{Sharing.} It is possible to add a new location to Google Maps. Sharing functionality is limited to the tool providing links and code for embedding.  

Evaluating Google Maps using our conceptual framework helped expose some gaps in its design, so we propose directions for future development. From the description above, it can be estimated that Google Maps' curation mechanisms lack some functionality for public and private curation. Improving public curation mechanisms introduces the possibility of channel-based discovery. Furthermore, adding category-guided navigation mechanisms can help with serendipitous discovery. By no means should an application like that be a replacement to Google Maps. However, it could be oriented towards social discovery and curation, as well as opportunistic place exploration, thereby complimenting the Google Maps application.  

} % end section 