\chapter{Web-based Information Discovery and Curation}
\label{chapter:chapter_related_work}

Given the complexity of Web-based information discovery and curation tasks, a variety of research topics are examined to gain an understanding of how current Web tools support these tasks, including information-related Web usage characteristics, current information behavior models, and other aspects of information discovery and curation.  This chapter outlines the key background literature that contributed to the development of the conceptual framework and helped answer the research questions. 

{\section{Information Behavior}
Information behavior refers to the totality of ways in which humans behave in relation to information~\cite{wilson2000human}.  A number of models and frameworks have attempted to represent human information behaviour in its entirety or to represent some of its components, such as information seeking and searching, information retrieval, information discovery, and information curation. 

One of the early information behavior models was proposed by Wilson~\cite{wilson1981user} in 1981. It states that information seeking behavior results from the user trying to satisfy a perceived information need, and consequently, the user makes demands on information systems. Success or failure of such demands dictates whether the process is repeated or, if the information need is satisfied, it is used or communicated with other people. 
\pagebreak

These underlying ideas remained in the revision of Wilson's model~\cite{wilson1997information} in 1997. In the new model, however, Wilson defined possible barriers (psychological, environments, demographic, etc.) that can impede information seeking. Additionally, the model recognizes that information seeking behaviour can take on many forms and is not limited to active searching. Saracevic~\cite{saracevic1996modeling} and Ingwersen~\cite{ingwersen1996cognitive} derived comparable models that focus on human behaviour when interacting with information retrieval systems. 

{\subsection{Information Seeking Models}
\textit{Information seeking} refers to ``the purposive seeking for information as a consequence of a need to satisfy some goal~\cite{wilson2000human}.'' A number of researchers have tried to identify what different modes of information seeking behaviour may entail. 

According to Kellar et al.~\cite{kellar2006goal}, information seeking is composed of browsing, fact finding, and information gathering. Although the authors categorized information gathering as part of information seeking, it appears to be more closely related to digital curation~\cite{beagrie2008digital,whittaker2011personal}. 

Bates~\cite{bates1986exploratory,bates2002toward} proposed a model of four information seeking modes: being aware, monitoring, browsing, and searching. Bates differentiated the modes based on the user's level of attention being active or passive, and information needs being directed or undirected. Thus, browsing can be characterized as undirected active information seeking because users do not know exactly what information they are looking for, but they are actively looking. Searching falls under active directed information seeking because the information need is clearly defined and the search is directed. Finally, monitoring and being aware are passive modes of information seeking although monitoring is directed and being aware is undirected.

Ellis et al.~\cite{ellis1989behavioural,ellis1993comparison,ellis1997modelling} proposed a model of information seeking characterized by six different patterns: starting, chaining, browsing, extracting, monitoring, and differentiating. Ellis' model complemented Kuhlthau's work, which correlated stages of information seeking with feelings, thoughts, actions, as well as anticipated information tasks~\cite{kuhlthau1991inside}.

Finally, Wilson also proposed a ``problem solving model'' of information seeking behavior~\cite{wilson1999models}. The model reflects on the idea that people engage in information seeking and searching in order to resolve some uncertainty that stands in the way of solving, defining, or identifying a problem.  
} % end subsection

{\subsection{Information Exploration}
\textit{Information exploration}, or exploratory search, does not have a single definition in the realm of information behavior. Waterworth highlights that exploration is a "broad" activity and identifies browsing as an example of exploration~\cite{waterworth1991model}. According to Marchionini~\cite{marchionini2006exploratory}, exploratory search involves learning (knowledge acquisition, comparison, comprehension, etc.) and investigating (analysis, synthesis, evaluation, discovery, etc.) Similar to Janiszewski ~\cite{janiszewski1998influence}, in terms of information exploration, my focus is on the visual aspects of information exploration, specifically visual and spatial data representations.
}
{\subsection{Information Foraging}
\textit{Information foraging theory} is another approach towards understanding how people adapt their strategies of interacting with technology when seeking, gathering, or consuming information, depending on the environment~\cite{pirolli1999information}. The theory resonates with explanations of human behavior in the context of food foraging. 

The underlying assumption of the information foraging theory is that people, similarly to when they forage for food, adopt their foraging strategies to the environment in order to gain the maximum amount of valuable information. The theory states that ``natural information systems evolve towards stable states that maximize gains of valuable information per unit cost.''

The theory introduces three key concepts to formulate an understanding of information foraging: information scent, information diet, and information patch. An \textit{information scent} refers to proximal cues (often visual or linguistic) that people use to identify the value of information. An \textit{information diet} deals with user preferences when it comes to information. At last, \textit{information patches} are clusters of information that an information system presents before the user. The theory with these concepts lays the foundation for existing information foraging models~\cite{fu2007snif,kitajima2000comprehension} as well as social information foraging models~\cite{pirolli2009elementary,fu2008microstructures}.  
}

{\subsection{Information Discovery}
Kerne and Smith proposed an information discovery framework~\cite{kerne2004information} that connects human cognitive processes or states to those of an information system. The framework represents a continuum of information flowing through different system and cognitive states as a result of an iterative reformulation process. The framework consists of five mental states: formulating a problem, evaluating results, updating and forming mental models, running mental models, and discovering solutions. Each mental state has a corresponding interaction with the system. For example, browsing resources (human-system interaction) facilitates evaluation or immediate results (cognitive state). The framework helps to understand the user's cognitive processes and provide affordances that facilitate information discovery.
}
   
{\subsection{Digital Curation}
In 2002, Bates extended her research on the topic of information behaviour with the notion of \textit{information farming}, which involves people collecting and organizing information for future use and revisitation~\cite{bates2002toward}. More commonly, information farming is referred to as digital curation. 

Wittaker believes that in terms of Web use, a significant shift is happening from information consumption to information curation. People no longer use the Web just to find and consume the information they are interested in, but they also try to save and manage that information so that it can be reaccessed and exploited later~\cite{whittaker2011personal}. 

Existing models and frameworks for information seeking, searching, exploration, discovery, and curation all try to explain human information-related behavior using different but comparable terminology. They help establish an understanding of how humans interact with information. However, many of them either fail to address required tool support for information-related activities or address it at a very high-level.  

{\section{Web Tasks and Modes of Web Use}
Outside the realm of cognitive models and frameworks for information behavior exists a body of research that examines information discovery, curation, and other Web information behaviours in terms of Web use and corresponding tasks, methods, and modes.

Kellar et al.~\cite{kellar2006goal} separated Web tasks into five categories: transactions, browsing, fact finding, information gathering, and other uncategorized tasks. In their later work, Kellar et al.~\cite{kellar2007field} added communication and maintenance as additional Web tasks. Similarly to Kellar et al., Sellen et al.~\cite{sellen2002knowledge} identified six tasks that are commonly performed by Web users: browsing, finding, housekeeping, information gathering, communicating, and transacting. Using different terms, Kellar et al. and Sellen et al. both identified highly comparable tasks, such as fact finding and finding [information], housekeeping and maintenance, etc. 

Building on Ellis' model of information seeking~\cite{ellis1989behavioural,ellis1993comparison,ellis1997modelling}, Choo et al.~\cite{choo2000information} derived anticipated Web tasks that correspond to the information seeking patterns in the model. According to the authors, when users \textit{identify} sources of interest, they usually identify which Websites can point to that information of interest.  \textit{Chaining} corresponds to users navigating through links on those initial pages. When people \textit{browse}, they scan top-level pages, headings, lists, and site maps. \textit{Differentiating} takes place when people bookmark, print, copy and paste information, or choose an earlier selected site. \textit{Monitoring} occurs when users revisit Web pages or receive updates from previously visited sites. Finally, \textit{extraction} can occur when the user systematically searches sites to extract information of interest.  

People often engage in information seeking activities to close some knowledge gap that occurred as a result of not having enough information to perform a task~\cite{proper1999information}. Therefore, when providing tool support for various information discovery tasks, it is useful to consider the motivations, as they can be different for each task. Morrison et al.~\cite{morrison2001taxonomic} make a distinction between methods of Web use and purposes. The authors derived a purpose-based taxonomy of Web use, including three purposes or motivations: finding information, comparing pieces of information or choosing products to make a decision, and using the Web to find relevant information to gain an understanding of some subject. Consequently, methods of finding information identified by Morrison et al. are collecting, finding, exploring, and monitoring. The differences between the two taxonomies suggest that different information seeking tasks may be performed to satisfy more than one information seeking purpose. Therefore, each purpose may require more than one task-supporting mechanism. 

Morrison et al. also draw a distinction between finding or looking up information and exploratory search. Whereas information lookup involves tasks such as fact retrieval, navigation, and verification, exploration is more cognitively demanding and involves learning and investigation~\cite{marchionini2006exploratory}. Learning and investigation can be performed over multiple iterations, and can involve learning though various media, "social searching", and serendipitous browsing performed with the goals of knowledge acquisition, socialization, forecasting, and planning.  
\pagebreak

Categorizing Web usage into information seeking, digital curation, and other Web tasks does not necessarily give full insight into how information-related tasks are performed. Lindley et al.~\cite{lindley2012s} conducted a qualitative study involving 24 participants, tracking their daily Web usage in the form of a diary. As a result of this study, the researchers identified five distinct modes of Web use: respite, orienting, opportunistic, purposeful, and lean-back. According to the authors, people browse the Web \textit{opportunistically} when they look for information related to some personal interest, long-term goal, or future ambition. \textit{Purposeful use} occurs when the users know what information they need to acquire or what online action they need to perform in order to continue or finish some other activity. \textit{Respite} mode usually occurs when users are in the process of waiting for something or taking a break, and it serves as a means for people to temporarily occupy themselves when high engagement with the content is not a requirement. \textit{Orienting} mode usually occurs when people want to be updated on what has been happening in their environment. Examples of this mode are checking email at work or looking at the news and updates on a social networking site. Finally, \textit{lean-back} mode of Web use can be thought of as listening to the radio or watching television, and usually involves watching videos online or browsing through other types of entertainment content. 

Lindley et al.'s primary motivations behind looking at use modes that occur when people browse the Internet were because that traditional Web use studies and Web tasks discovered by other researchers do not reflect the depth of user's intentions online. Understanding the characteristics of different modes guides the design of Web interaction. For example, opportunistic use can have unarticulated or continuously changing information needs. People often cannot indicate the completion of Web tasks, and they finish whenever they have been browsing the Internet for too long, or whenever they need to complete some other task of higher priority. Then, they will often resume their opportunistic information seeking. Finally, opportunistic use is `grasshopper-like', which means that users jump from one resource to another~\cite{lindley2012s}. From these factors, we can assume that to support such Web usage, we would need to consider mechanisms for supporting users' information needs, revisitation, and arbitrary navigation.

\pagebreak
Different taxonomies of information seeking and curation tasks reflect on the actual Web usage rather than theoretical modeling of human behavior. However, these taxonomies still focus on human activities when they interact with technology. A better understanding of how the system can support these activities is needed in order to effectively support human information-related interactions. 

} % end subsection

{\section{Collaborative Information Discovery and Curation}
By surveying 204 Web users, Morris found that people often desire to or do collaborate on information seeking tasks~\cite{morris2008survey}. To collaborate on information seeking, people often use instant messaging, email, create documents and Webpages to share information. Occasionally, collaborative information seeking occurs when collaborators work side by side and share search results in person.

Collaborative information-related activities on the Web are not limited to information seeking. Collaborative information tagging is a way of organizing content for future search and navigation. Although it is usually performed for personal reasons, tagging greatly enhances information retrieval~\cite{golder2006usage}.

} % end subsection

{\section{Summary}
Today, there are a multitude of tools that support different aspects of information discovery and curation, but understanding how these tools are similar (or differ) is difficult. Moreover, the existing research is not useful for identifying gaps in current tools or ways that current tools may be improved to support information discovery and curation. I address these problems by presenting a conceptual framework for information discovery and curation (see Chapter~\ref{chapter:framework}).
}
