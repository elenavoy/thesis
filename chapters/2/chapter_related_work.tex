\chapter{Related Work}
\label{chapter:chapter_related_work}

{\section{\sloppy Web-based Information \\Discovery and Curation}

Several researchers have studied various aspects of Web-based information discovery. To gain an understanding of how current Web tools support information discovery and curation, we first studied known characteristics of information-related Web usage, including high-level Web tasks, information seeking behavior, information curation, collaboration in information seeking, and modes of Web use.  

{\subsection{Web Tasks}

Kellar et al. ~\cite{kellar2006} separated Web tasks into five categories: transactions, browsing, fact finding, information gathering, and other uncategorized tasks, with information seeking being composed of browsing, fact finding, and information gathering. Although the authors categorized information gathering as part of information seeking, it is in fact more closely related to digital curation ~\cite{beagrie, wittaker}. In their later work, Kellar et al. ~\cite{kellar2007} added communication and maintenance as additional Web tasks. 

Similarly to Kellar et al., Sellen et al. ~\cite{sellen} identified six tasks that are commonly performed by Web users: browsing, finding, housekeeping, information gathering, communicating, and transacting. Therefore, Kellar et al. and Sellen et al. both identified browsing, fact finding, and information gathering as information-related tasks that users perform online.   

People often engage in information seeking activities to close some knowledge gap that occurred as a result of not having enough information to perform a task ~\cite{proper}. Therefore, when providing tool support for various information discovery tasks, it is useful to consider the motivation behind these tasks as it can be different for each task. Morrison et al. ~\cite{morrison} make a distinction between methods of Web use and purposes. The authors derived a purpose-based taxonomy of Web use, including three purposes or motivations: finding information, comparing pieces of information or choosing products to make a decision, and using the Web to find relevant information to gain an understanding of some subject. Consequently, methods of finding information identified by Morrison et al. are collecting, finding, exploring, and monitoring. The differences between the two taxonomies suggest that different information seeking tasks may be performed to satisfy more than one information seeking purpose. Therefore, each purpose may require more than one task-supporting mechanism. 

Morrison et al. also draws distinction between finding or looking up information and exploratory search. Whereas information lookup involves tasks such as fact retrieval, navigation, and verification, exploration is more cognitively demanding and involves learning and investigation ~\cite{marchionini}. Learning and investigation can be performed over multiple iterations, and can involve learning though various media, "social searching", and serendipitous browsing performed with the goals of knowledge acquisition, socialization, forecasting, and planning.  

} % end subsection

{\subsection{Information Behavior Models}

A number of researchers have proposed models of information seeking and information behavior. Wilson ~\cite{wilson1999} summarized some of the key work ~\cite{ellis1989, dervin, kuhlthau, wilson1997, wilson1981} on information behavior and proposed a new model. According to Wilson's original model of information behavior ~\cite{wilson1981}, information seeking behavior results form the user trying to satisfy their perceived information need. Consequently, the user makes demands on information systems. Success or failure of such demands dictates whether the process is repeated or, if the information need is satisfied, used or communicated with other people. In addition, Wilson defined possible barriers that can impede information seeking behaviors, as well as context that influences formation of the information need. These underlying ideas remained in the revision of Wilson's model ~\cite{wilson1997}. Finally, Wilson proposed a "problem solving model" of information seeking behavior. The model reflects on the idea that people engage in information seeking and searching in order to resolve some uncertainty that stands in the way of solving, defining, or identifying a problem.    

Ellis et al. ~\cite{ellis1989, ellis1997, ellis1993} proposed a model of information seeking characterized by six different patterns: starting, chaining, browsing, extracting, monitoring, and differentiating. Subsequently, Choo et al. ~\cite{choo} derived anticipated Web tasks that correspond to these patterns. According to the authors, when users identify sources of interest, they usually identify which Websites can point to that information of interest.  Chaining occurs when users navigate through links on those initial pages. When people browse, they scan top-level pages, headings, lists, and site maps. Differentiating takes place when people bookmark, print, copy and paste information, or choose an earlier selected site. Monitoring occurs when users revisit Web pages or receive updates from previously visited sites. Finally, extraction can occur when the user systematically searches sites to extract information of interest. Ellis' model also complemented Kuhlthau's ~\cite{kuhlthau} work which corresponded stages of information seeking with feelings, thoughts, actions, as well as anticipated information tasks.

Information retrieval behaviors are further studied by Saracevic ~\cite{saracevic} and Ingwersen ~\cite{ingwersen} who derived models concentrating on cognitive processes of information seeking. 
 
Bates ~\cite{bates1986} proposed a model of four information seeking modes: being aware, monitoring, browsing, and searching. Bates differentiated the modes based on the user's level of attention being active or passive, and information needs being directed or undirected. Thus, browsing can be characterized as undirected active information seeking because users do not know directly what information they are looking for, but they are actively looking. Searching falls under active directed information seeking because the information need is clearly defined and the search is directed. Finally, monitoring and being aware are passive modes of information seeking although monitoring is directed and being aware is undirected.

} % end subsection
   
{\subsection{Digital Curation}

In 2002, Bates ~\cite{bates2002} extended her research with the notion of information farming, which involves people collecting and organizing information for future use and revisitation. More commonly, information farming is referred to as digital curation, which is the notion of collecting and managing digital information for the purpose of adding value to the collection and revisitation ~\cite{beagrie}. Wittaker ~\cite{wittaker} believes that in terms of Web use, a significant shift is happening from information consumption to information curation, which means that people no longer just use the Web to find and consume the information that they are interested in, but they also try to save and manage that information so that it can be reaccessed and exploited later. 

{\subsection{Collaboration and Information Seeking}
By surveying 204 Web users, Morris found that people often desire to or do collaborate on information seeking tasks ~\cite{morris}. To collaborate on information seeking, people often use instant messaging, email, and create documents and Webpages to share information. Occasionally, collaborative information seeking occurs when collaborators work side by side and share search results in person.

Collaborative information-related activities on the Web are not limited to information seeking. Collaborative information tagging is a way of organizing content for future search and navigation. Although it is usually performed for personal reasons, tagging greatly enhances information retrieval ~\cite{golder}.

} % end subsection
{\subsection{Modes of Web Use}
Categorizing Web usage into information seeking, digital curation, and other Web tasks does not necessarily give full insight about how information-related tasks are performed. Lindley et al. ~\cite{lindley} conducted a qualitative study involving 24 participants, tracking their daily Web usage in the form of a diary. As a result of this study, the researchers identified five distinct modes of Web use: respite, orienting, opportunistic, purposeful, and lean-back. According to the authors, people browse the Web \textit{opportunistically} when they look for information related to some personal interest, long-term goal, or future ambition. \textit{Purposeful use} occurs when the users know what information they need to acquire or what online action they need to perform in order to continue or finish some other activity. \textit{Respite} mode usually occurs when users are in the process of waiting for something or taking a break, and it serves as a means for people to temporarily occupy themselves when high engagement with the content is not a requirement. \textit{Orienting} mode usually occurs when people want to be updated on what has been happening in their environment. Examples of this mode are checking email at work or looking at the news and updates on a social networking site. Finally, \textit{lean-back} mode of Web use can be thought of as listening to the radio or watching TV, and usually involves watching videos online or browsing through other types of entertainment content. 

Lindley et al.'s primary motivations behind looking at use modes that occur when people browse the Internet was that traditional Web use studies and Web tasks discovered by other researchers cannot reflect the depth of user's intentions online. Understanding the characteristics of different modes guides the design of Web interaction. For example, opportunistic use can have blurry and continuously changing information needs. People often cannot indicate the completion of Web tasks, and they finish whenever they have been browsing the Internet for too long, or whenever they need to complete some other task of higher priority. Then, they will often resume their opportunistic information seeking. Finally, opportunistic use is 'grasshopper-like', which means that users jump from one resource to another ~\cite{lindley}. From these factors, we can assume that to support such Web usage, we would need to consider mechanisms for supporting users' information needs and support revisitation and arbitrary navigation.

} % end subsection

Today, there are a multitude of tools that support different aspects of information exploration and curation, but understanding how these tools are similar (or differ) is difficult. Moreover, the existing research is not useful at helping identify gaps in current tools or ways that current tools may be improved to support information
exploration and curation. Thus, we present a framework of Web application design factors and questions that facilitate information discovery and curation (see Sec. 4).
} % end section