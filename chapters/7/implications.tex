\chapter{Research and Design Implications}
\label{chapter:implications}

The conceptual framework for information discovery and curation is designed to perform formative and summative evaluation of existing Web applications and to reveal how these tools support information-related activities in question. The framework as a tool and its ability to guide the process of analyzing Web applications makes it broadly applicable in research and Web design. 

In Chapter~\ref{chapter:evaluation}, I demonstrated how the framework can be used to reveal missing features in tools. Using similar methods, the framework can also be applied to compare different Web applications. When used for evaluation, the framework helps to identify which areas of a tool require further attention. Therefore, the framework can be helpful for designers who wish to improve existing tools or get ideas for new information discovery and curation applications. 

Factors and questions of the framework are there to guide the developer, but they do not dictate which features should be in an application. In other words, the framework helps expose gaps, but it is up to designers to decide whether those gaps need to be closed. In fact, some gaps cannot be closed because of certain constraints, such as data type and system design.

User interface designers face certain trade-offs when developing applications. Therefore, it is not always advantageous to implement all missing features. For example, providing the support to customize spatial arrangement of multiple resources can undermine the consistency of their representation. 

\pagebreak

In the research domain, the framework can serve as a guide for drawing distinctions between different Web-based information discovery and curation applications, finding gaps in tools that can be studied, and selecting cases for studies based on required functionality. Hence, both researchers and developers can benefit from the systematic tool examination guided by the framework.

Even though applying the framework requires initial expertise and critical reasoning, it opens up opportunities for research and practice. Systematic evaluation of Web tools for information discovery and curation helps the designer improve user experience and gain better understanding of information behaviour within a given system. 




