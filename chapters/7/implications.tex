\chapter{Implications}
\label{chapter:implications}

{\section{Research and Design \\ Implications}
In the previous section, we demonstrated how the framework can be used to reveal missing features in tools. We also showed how the framework can be helpful for designers who wish to improve existing tools or get ideas for new information discovery applications. 

Factors and questions of the framework are there to guide the developer, but they do not dictate which features should be in the application. In other words, the framework helps expose gaps, but it is up to designers to decide whether those gaps need to be closed---some gaps cannot be closed because of certain constraints, such as data type and system design.

As with the Google Maps example, designers face certain trade-offs when developing applications with the help of the framework. For example, high precision with navigation mechanisms can potentially eliminate some opportunities for serendipitous discovery. 

In the research domain, the framework can serve as a guide for selecting cases for studies and drawing distinctions between different Web-based information seeking applications. Hence, both researchers and developers can benefit from the systematic tool exploration guided by the framework.

} % end section