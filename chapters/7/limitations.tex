\chapter{Limitations}
\label{chapter:limitations}

{\section{Limitations}
The case study we conducted has a number of limitations: a lack of documentation, literature, and formal descriptions of available features for some applications introduces a threat to construct validity of the study. In addition, information discovery tools and features can be used in manners unintended or unforeseen by designers and developers. Therefore, the use of some features within information discovery applications was recorded based on the researchers' interpretations. To compensate for such limitations, the researchers employed the tools for personal use over an extended period of time to gain a deeper understanding of their use. In addition, the researchers considered some cases with repeating functionality and design to be able to validate or clarify prior findings. 

Many Web applications rapidly evolve. Therefore, our tool analysis only applies to tools at the moment of our study.

Only Web applications running in browsers on a desktop computer were considered in this study. Our study can be extended with use of various devices, such as smartphones and tablets, as information discovery patterns and mechanisms may very for different platforms. 

Another limitation was the lack of prior research studies on the subject matter. Some researchers have studied information seeking models and high-level Web tasks, but there is a lack of literature on how to enable and support different Web tasks. This opens up opportunities for future research to analyze methods of developing and building frameworks for facilitating and evaluating tools that support other Web tasks, such as communication, transactions, and goal realization.

} % end section