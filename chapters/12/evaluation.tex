\chapter{Framework Validation}
\label{chapter:evaluation}

In order to validate the conceptual framework (see Chapter~\ref{chapter:framework}) and verify its stability, I applied it to evaluate five of the applications that were used in the construction of the preliminary framework (see Chapter~\ref{chapter:old_framework}): Pinterest, Google Maps, Wikipedia, Delicious, and Yelp. In this chapter, I summarize evaluations of the tools and propose directions for development. 

{\section{Pinterest}
Pinterest is a Web application designed for image discovery and curation, which are oriented towards finding inspiration and collecting knowledge about hobbies and interests.  Users of Pinterest are commonly referred to as `pinners'. Resources on Pinterest are called `pins', and each `pin' consists of an image, short description, user's name, and name of a collection that the pin belongs to. More information is available once the user clicks on one of those `pins'.

Motivated by the desire to gain inspiration and knowledge, as noted above, users of Pinterest have either underdefined or absent information needs. Other motives for using Pinterest, that follow from the main purpose of the tool, could be to rediscover previously found information (and possibly use it), to be oriented about new `pins' that emerge from subscribed channels, and to collect information for the purposes of future rediscovery and collection itself.

Navigation in Pinterest is mostly supported by descriptional, referential, or system-regulated mechanisms. Although an explicit \textbf{opportunistic navigation mechanism} is absent, both descriptional and referential mechanisms usually return novice and serendipitous results to facilitate opportunistic browsing. Descriptional navigation is enabled with a \textbf{guided search} mechanism, which suggests search terms to the user. 

Referential navigation is enabled in Pinterest using a range of techniques. To support articulation of an information need, \textbf{category-based} navigation mechanism makes further suggestions on subcategories, or interests. Through clicking on any of the `pins', the user can see related resources, which enables \textbf{resource-based} referential navigation. Most of the images in Pinterest are `pinned' from other Websites, and users are provided with links to their original sources. Therefore, Pinterest supports \textbf{integrated} referential navigation.

System-regulated navigation within Pinterest is highly personalized. When the user enters the site, they see the history of their own information gathering activities and updates from the people they are subscribed to. Additionally, the application suggests \textbf{featured} `pins' selected based on the user's personal interests.

To reinforce discovery of visual data, Pinterest extensively supports various exploration mechanisms. Multiple resources are represented in a \textbf{gallery layout} which is often referred to as a 'pinboard'. Such layout provides good spatial support for exploration and makes it easier to build a mental model of the tool by drawing analogies with a real pinboard. Users can create multiple `pinboards' (also known as `boards'), and they can \textbf{personalize} covers of all `boards' to enhance future exploration and rediscovery.

A single resource does not have a lot of distinct spatial arrangements; however, visually it provides a glimpse into what can be found on the Website that the image came from, with \textbf{textual preview} being limited to the address of the Website. 

Information management is accomplished through sorting `pins' into different collections (`pinboards') thus enabling \textbf{collection-based classification} and \textbf{internal preservation of internal and external resources}. All of user information collecting actions are \textbf{automatically} preserved and displayed. Users can augment the information pool by uploading new `pins', commenting on existing `pins', or adding descriptions. Users can also \textbf{internally share} 'pins' among themselves. Channel-picking actions are carried out through following or \textbf{subscribing} to users or individual `pinboards'. The system also `automatically' sends `notifications' though emails and `suggests' new `boards' to follow.

Applying the framework to evaluate Pinterest revealed that the tool employs a  variety of techniques to facilitate information discovery and curation. However, individual mechanisms could be further improved. For example, \textbf{textual previews} of multiple and individual resources is rather limited, and provides little insight to what information source Websites actually contain. As another example, Pinterest could benefit from \textbf{automatically classifying} `pins' into `boards' because finding an appropriate `board' for a `pin' can be difficult with a large number of existing `boards'. Overall, Pinterest provides rich support for information disocvery and curation, and in some ways, enable each of the discovery or curation actions of the conceptual framework. 
} % end section


{\section{Google Maps}
Google maps is a Web application oriented towards navigation and place discovery. It provides services for finding directions to places, their addresses, and other information. By answering the questions from the conceptual framework for information discovery and curation and analyzing the application, I get the following description of Google Maps.

The primary motive behind using Google Maps is usually to a find specific information, most commonly directions. The information need can be either very precise, such as looking for an address of a particular place, or it can be slightly more ambiguous, such as looking for a coffee shop within certain area. sometimes, users also return to the site to rediscover previously found directions or addresses.

Information discovery in Google Maps is usually initiated by \textbf{search}, and thus, the user needs to formulate their information need---the application lacks some of the internal referential navigation mechanisms, so there is nothing that aids users in this task. The one type of referential navigation that Google Maps does support is resources-based. For example, the user can click on ``Search nearby" suggested link to find places near another place. Google Maps is conveniently \textbf{integrated} with Google+, allowing access to relevant information, such as reviews, images, and hours of operation, and thus, enabling resource-based integrated referential navigation. Search-based navigation within Google Maps is usually precise and returns accurate search results for specific places making it easy to rediscover information using search. 

Google Maps lacks \textbf{opportunistic navigation mechanisms}, and it provides limited support for system-regulated navigation by displaying \textbf{personalized featured content} in a form of a map of the user's location.  

Considering the nature of Google Maps, the semantic of the spatial arrangement of resources is defined by the locations of actual places on the map. More information is presented as a \textbf{list}. \textbf{Consistency} in how resources are represented makes it easy to find information, such as addresses and contacts. Furthermore, multiple and individual resources provide \textbf{visual previews} that show photographs added by users or street views respectively.

Google Maps supports curation mainly through personal preservation. Users can only bookmark places to the "My Places" list --- by adding \textbf{internal content to internal storage}. Other types of place preservation are possible through Google+, however, not within Google Maps. Users can also \textbf{evaluate} and \textbf{annotate} places through Google+, and aggregated reviews and ratings are visible on Google Maps. Sharing is enabled through providing functionality to add new locations to Google Maps and supplying links and code for embedding.  

Channel-picking actions are usually enabled within applications with frequently updating content. Content provided by Google Maps is fairly stable, and therefore, there are no channel-picking mechanisms used by the application.

Evaluating Google Maps using our conceptual framework helped expose some gaps in its design, so I propose directions for future development. From the description above, it can be estimated that Google Maps' curation mechanisms lack some functionality for public and private curation. Improving public curation mechanisms as well as adding functionality for channel-picking introduces the possibility of channel-based discovery. By no means should an application like that be a replacement to Google Maps. However, it could be oriented towards social discovery and curation, as well as channel-based discovery, thereby complimenting the Google Maps application.  

} % end section

{\section{Wikipedia}
Wikipedia is an open encyclopedia containing millions of articles contributed by people from all over the world. The primary motive for discovering information on Wikipedia is to gain knowledge to either answer a specific question or to learn more about a general topic, such as art or history.

Wikipedia supports a wide range of navigational mechanisms. Descriptional navigation on Wikipedia is accomplished through \textbf{search}. Results, however, are not \textbf{personalized to the user}, and the search mechanism is not \textbf{guided} by suggestions of what search terms to use. Referential mechanisms include a \textbf{categories} which consist of broad topics and return \textbf{featured articles}. Not all Wikipedia articles are \textbf{integrated} with other Websites and Web applications; however occasionally, articles contain ``External Links" section that provides links to external resources. Opportunistic navigation provides opportunity to serendipitously discover new articles, and it is enabled though ``Random article'' link located on the navigation sidebar.

Wikipedia directly displays featured articles, but the content is, again, not tailored to the user. Wikipedia updates featured content that can be viewed on the front page and when navigated to using categories. The users can also see history of recently updates content. 

Exploration of multiple resources is limited to when the target of the search query is unclear. Then, search results are presented in a \textbf{list} layout, where links are represented as text. 

There is not preservation mechanism available at the site, as well as no management mechanisms. Augmentation is possible through personal contribution to the content of articles.  Sharing of Wikipedia articles is also not supported. 

Information on Wikipedia is publicly curated by thousands of users, who augment existing articles and \textbf{add} new. Although it is not possible to \textbf{subscribe} to any particular channel, 

Evaluation of Wikipedia revealed that the major gaps in its discovery and curation support relate to personal curation and visual exploration of multiple resources. Adding more cognitive support, such as suggesting search terms or topics of interest can also improve user experience.

} % end section


{\section{Delicious}
Delicious is a bookmarking application. 

In delicious, users can navigate using search (descriptional navigation). Referential navigation is not as developped.

Exploration for the most part is limited to non-visual factors.

Delicious is integrated with many other Websites through linking. Visual linking is only provided in the "Trending" section of the tool.

Curation is a very important aspect of Delicious. Management is performed through tagging. 

Delicious enables internal preservation of external resources. External sharing of internal resources, as well as adding new resources. 

Information augmentation is possible through tagging and commenting on the added links.

Channeling is performed thorough subscription mechanisms: users can subscribe to other users and create their networks. 
"Trending" section displays results of social curation, and therefore, enables channel-based discovery.

According tot he framework, Delicious lacks visual exploration mechanisms, and certain personalization mechanisms. However, the "Trending" part of the system does provide visual previews and arrange resources in a grid. Although it helps with visual and spatial exploration, at the same time it undermines consistency of resource representation. 

} % end section

{\section{Yelp}
Yelp is a Web application used to discover local businesses. 

Descriptional navigation  is again supported with search-based navigation. 

Referential navigation is enabled using category-based navigation.

Both visual and spatial exploration is enabled at the site. 

Resources are integrated with Google Maps and with the business Websites. In case of Google Maps, visual integration is applied but not with other links.

Users can bookmark businesses they like within the system. 

Evaluation is possible by rating the businesses. Reviews can also be evaluated by choosing between 'Useful', 'Funny', and 'Cool' metrics.

Users can add resources by writing reviews which also contributes to information augmentation. Users can also add images of the business. 

On Yelp, the user can see latest updates from activities of other users. 

Identified gaps include lack of management mechanisms when businesses are bookmarked.
} % end section

{\section{Discussion}
The evaluations of existing Web tools, provided in this chapter, using the conceptual framework for information discovery and curation demonstrate applicability of the framework. A set of questions, provided by the framework, can help in the process of tool evaluation, and they can also be applied in drawing distinctions between different tools. 

The framework associates different information discovery and curation actions with concrete mechanisms. However, it is not always clear which actions in the framework the tool needs to support. With help of the framework, some of the requirements (but not all) can be derived from the analysis of other applications, which might be in a similar domain or possess desired properties. 

Another source that can hint on which actions need support is the motive for discovery and curation activities in an application. For example, if the motive is to discover information with an undefined information need, the application  can either be tailored to support serendipitous discovery by providing opportunistic navigation mechanisms, or it can help the user to formulate the information need by suggesting search terms and categories.

Finally, the need for a given discovery or curation-supporting mechanism can be evaluated using intuition and experience of the designer. In some cases, it is especially difficult to estimate the importance of a mechanism in a specific application using known measures; however, as with many other decisions when it comes to developing or improving an application, the designer can use their judgment along with subsequent studies and evaluation. 

Evaluation and comparison of different Web applications can reveal useful insights about the mechanics of how the system induces user experience, and they can expose certain unadressed needs. The next Chapter illustrates the design process on an application that emerged from evaluation of other tools using the conceptual framework. 

} % end section