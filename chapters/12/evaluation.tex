\chapter{Framework Validation}
\label{chapter:evaluation}

In order to validate the conceptual framework (see Chapter~\ref{chapter:framework}) and verify its stability, I applied it to the evaluation of five of the applications that were used in the construction of the preliminary framework (see Chapter~\ref{chapter:old_framework}): Pinterest, Google Maps, Wikipedia, Delicious, and Yelp. For each of the Web applications, I first summarize my observations resulting from asking the questions from the framework in a systematic manner. Based on my assumptions and judgment and with help of the framework, I propose directions for future development and reflect on the necessity of having some of the mechanisms, as not all mechanisms are always needed.  

{\section{Pinterest}
Pinterest is a Web application designed for image discovery and curation, which is oriented towards finding inspiration and collecting knowledge about hobbies and interests~\cite{gilbert2013need,zarro2012pinterest,ottoni2013ladies}.  Users of Pinterest are commonly referred to as `pinners'. Resources on Pinterest are called `pins', and each `pin' consists of an image, short description, user's name, and name of a collection that the pin belongs to. More information is available once the user clicks on one of those `pins'.

Motivated by the desire to gain inspiration and knowledge, as noted above, users of Pinterest have either underdefined or absent information needs. Other motives for using Pinterest, that follow from the main purpose of the tool, could be to rediscover previously found information (and possibly use it), to be oriented about new `pins' that emerge from subscribed channels, and to collect information for the purposes of future rediscovery and collection itself.

Navigation in Pinterest is mostly supported by descriptional, referential, or system-regulated mechanisms. Although an explicit \textbf{opportunistic navigation mechanism} is absent, both descriptional and referential mechanisms usually return novel and serendipitous results to facilitate opportunistic browsing. Descriptional navigation is enabled with a \textbf{guided search} mechanism, which suggests search terms to the user. 

Referential navigation is enabled in Pinterest using a range of techniques. To support articulation of an information need, \textbf{category-based} navigation mechanism makes further suggestions on subcategories, or interests. Through clicking on any of the `pins', the user can see related resources, which enables \textbf{resource-based} referential navigation. Most of the images in Pinterest are `pinned' from other Websites, and users are provided with links to their original sources. Therefore, Pinterest supports \textbf{integrated} referential navigation.

System-regulated navigation within Pinterest is highly personalized. When the user enters the site, they see the history of their own information gathering activities and updates from the people they are subscribed to. Additionally, the application suggests \textbf{featured} `pins' selected based on the user's personal interests.

To reinforce discovery of visual data, Pinterest extensively supports various exploration mechanisms. Multiple resources are represented in a \textbf{gallery layout} which is often referred to as a 'pinboard'. Such layout provides good spatial support for exploration and makes it easier to build a mental model of the tool by drawing analogies with a real pinboard. Users can create multiple `pinboards' (also known as `boards'), and they can \textbf{personalize} covers of all `boards' to enhance future exploration and rediscovery.

A single resource does not have a lot of distinct spatial arrangements; however, visually it provides a glimpse into what can be found on the Website that the image came from, with \textbf{textual preview} being limited to the address of the Website. 

Information management is accomplished through sorting `pins' into different collections (`pinboards') thus enabling \textbf{collection-based classification} and \textbf{internal preservation of internal and external resources}. All of user information collecting actions are \textbf{automatically} preserved and displayed. Users can augment the information pool by uploading new `pins', commenting on existing `pins', or adding descriptions. Users can also \textbf{internally share} 'pins' among themselves. Channel-picking actions are carried out through following or \textbf{subscribing} to users or individual `pinboards'. The system also `automatically' sends `notifications' though emails and `suggests' new `boards' to follow.

Applying the framework to Pinterest revealed that the tool employs a  variety of techniques to facilitate information discovery and curation. However, individual mechanisms could be further improved. For example, \textbf{textual previews} of multiple and individual resources is rather limited, and provides little insight to what information source Websites actually contain. As another example, Pinterest could benefit from \textbf{automatically classifying} `pins' into `boards' because finding an appropriate `board' for a `pin' can be difficult with a large number of existing `boards'. Overall, Pinterest provides rich support for information disocvery and curation, and in some ways, enable each of the discovery or curation actions of the conceptual framework. 
} % end section


{\section{Google Maps}
Google maps is a Web application oriented towards navigation and place discovery~\cite{gibson2006google}. It provides services for finding directions to places, their addresses, and other information. By answering the questions from the conceptual framework for information discovery and curation and analyzing the application, I arrive at the following description of Google Maps.

The primary motive behind using Google Maps is usually to find specific information about a place, most commonly directions to that place. The information need can be either very precise, such as looking for an address of a particular place, or it can be slightly more ambiguous, such as looking for a coffee shop within a certain area. Sometimes users also return to the site to rediscover previously found directions or addresses.

Information discovery in Google Maps is usually initiated by a \textbf{search}, and thus, the user needs to formulate their information need---the application lacks some of the internal referential navigation mechanisms, so there is nothing that aids users in this task. The one type of referential navigation that Google Maps does support is \textbf{resource-based}. For example, the user can click on the ``Search nearby" suggested link to find places near another place. Google Maps is conveniently \textbf{integrated} with Google+, allowing access to relevant information, such as reviews, images, and hours of operation, and thus, enabling resource-based integrated referential navigation. Search-based navigation within Google Maps is usually precise and returns accurate search results for specific places, making it easy to rediscover information using search. 

Google Maps lacks \textbf{opportunistic navigation mechanisms}, and it provides limited support for system-regulated navigation by displaying \textbf{personalized featured content} in a form of a map of the user's location.  

Considering the nature of Google Maps, the semantic of the spatial arrangement of resources is defined by the locations of actual places on the map. More information is presented as a \textbf{list}. \textbf{Consistency} in how resources are represented makes it easy to find information, such as addresses and contacts. Furthermore, multiple and individual resources provide \textbf{visual previews} that show photographs added by users or street views respectively.

Google Maps supports curation mainly through personal preservation. Users can only bookmark places to the "My Places" list --- by adding \textbf{internal content to internal storage}. Other types of place preservation are possible through Google+, however, not within Google Maps. Users can also \textbf{evaluate} and \textbf{annotate} places through Google+, and aggregated reviews and ratings are visible on Google Maps. Sharing is enabled through providing functionality to add new locations to Google Maps and supplying links and code for embedding.  

Channel-picking actions are usually enabled within applications with frequently updating content. Content provided by Google Maps is fairly stable, and therefore, there are no channel-picking mechanisms used by the application.

Evaluating Google Maps using my conceptual framework also exposes some gaps in its design. From the description above, it can be estimated that Google Maps' curation mechanisms lack some functionality for public and private curation. Improving public curation mechanisms as well as adding functionality for channel-picking introduces the possibility of channel-based discovery. By no means should an application like that be a replacement to Google Maps. However, it could be oriented towards social discovery and curation, as well as channel-based discovery, thereby complimenting the Google Maps application.  

} % end section

{\section{Wikipedia}
Wikipedia is an open encyclopedia containing millions of articles contributed by people from all over the world~\cite{kittur2007power,volkel2006semantic}. The primary motive for discovering information on Wikipedia is to gain knowledge to either answer a specific question or to learn more about a general topic, such as art or history.

Wikipedia supports a wide range of navigational mechanisms. Descriptional navigation on Wikipedia is accomplished through \textbf{search}. Results, however, are not \textbf{personalized to the user}, and the search mechanism is not \textbf{guided} by suggestions of what search terms to use, which could help the user to formulate their information need. Referential mechanisms include a \textbf{categories} which consist of broad topics and return \textbf{featured articles}. Not all Wikipedia articles are \textbf{integrated} with other Websites and Web applications; however occasionally, articles contain ``External Links" section that provides links to external resources. 

Opportunistic navigation facilitates serendipitous discovery of new articles, and it is enabled though ``Random article'' link located on the navigation sidebar. Wikipedia regularly updates \textbf{featured content} that can be viewed on the front page and when navigated to using categories. Users can also see the history of recently updates articles. 

Exploration of multiple resources is limited to when the target of the search query is unclear. Then, search results are presented in a \textbf{list} layout, where links are represented as \textbf{text}. Single resource exploration mechanisms include a table of contents in large articles, which can serve as a referential navigation mechanism as well as \textbf{textual cue} of what the article consists of, and occasional \textbf{visual} material that aids in communicating ideas of the article.  

Information on Wikipedia is publicly curated by thousands of users, who improve existing articles and \textbf{add} new. Although it is not possible to \textbf{subscribe} to any particular channel, moderators of Wikipedia regularly update featured articles and information. Augmentation is possible through personal contribution to the content of articles. However, there are no private curation mechanisms that could be used for personal benefit. 

The application of the framework to Wikipedia revealed that major gaps in its discovery and curation support are related to personal curation and visual exploration of multiple resources. For example, since there are no mechanisms for personal preservation and management, users cannot build their own knowledge maps and continuously engage with the content. Adding more cognitive support or personalization, such as suggesting search terms or topics of interest could perhaps also improve user experience.

} % end section


{\section{Delicious}
Delicious is a Web application designed for social bookmarking and information discovery~\cite{rader2008influences, tesconi2008semantify}. The primary motive for using Delicious is to preserve articles found on other Websites for future access and to discover new articles or blog posts. When used for discovery, the information need is usually underdefined or absent unless the user's motive is to rediscover previously found information.   

In Delicious, users can navigate using \textbf{search} (descriptional navigation). Referential navigation within the application is accomplished through \textbf{resource-based} search, which returns related links. Delicious is also integrated with many other Websites through linking, which enables \textbf{integrated referential navigation}. The application does not support opportunistic browsing, but it does provide an option for system-regulated browsing through the ``Trending'' section of the Website which displays \textbf{featured content} based on article popularity. The ``Trending" section displays results of the social curation, and therefore, enables channel-based discovery. Since the ``Trending" section displays results of social curation, it enables channel-based discovery.

Exploration of a single resource is not enabled in Delicious, and the mechanisms for exploring multiple resources, for the most part, are limited to \textbf{textual previews} and a \textbf{list} layout. However, the ``Trending" section of the system does provide \textbf{visual previews} and arrange resources in a \textbf{grid} layout. In addition, it provides extended \textbf{textual previews} or snippets of corresponding articles, making it easier to follow the information scent when navigating across various Websites. Although these mechanisms help with visual and spatial exploration, having them applied to only one section of Delicious simultaneously undermines the \textbf{consistency} of multiple resource representation. 

Since the primary motive for using Delicious is to preserve and share information, support for curation is the core feature of this Web application. Management can be performed through \textbf{tagging}, and the system \textbf{suggests} tags based on the tag cloud. Delicious supports \textbf{internal preservation of external and internal resources}, \textbf{external sharing of internal resources}, as well as \textbf{adding} new resources. Information augmentation is possible through commenting on (or \textbf{annotating}) added links. Channel-picking is performed thorough \textbf{subscription mechanisms} --- users can follow other users and build their networks. 

According to the framework, Delicious lacks extensive visual exploration mechanisms in most of its sections. Since Delicious is designed to facilitate article discovery, and Web page titles often provide limited insight about an article, the tool could benefit from providing \textbf{textual} and \textbf{visual} article previews in all of its sections to help the user follow the information scent. 
} % end section

{\section{Yelp}
Yelp is a Web application used to discover local businesses, such as restaurants, beauty salons, and shops~\cite{luca2011reviews}. The primary motive for discovering information on Yelp is to evaluate and compare businesses in certain domains and geographical locations. Therefore, users either have defined information needs, such as rating of a specific business, or underdefined information needs, such as a good restaurant in a certain area. Most of the content on Yelp consists of user reviews and business evaluations or ratings. 

Descriptional navigation in Yelp is once again supported using the \textbf{search feature}, which does not only \textbf{suggest} search terms to the user, but also allows to further specify the proximate location of a business. Referential navigation is enabled using \textbf{category-based} navigation and \textbf{filtering}. \textbf{Integrated references} are provided to navigate to Google Maps and to business Websites.  On Yelp, the user can see a \textbf{news feed} of activities of other users, as well as \textbf{featured} businesses tailored to the user based on their location. 

Both visual and spatial exploration mechanisms are enabled on Yelp. Visual explorations of multiple and single resources are facilitated by numerous photographs, icons, maps, and visuals depicting ratings. Spacial representation of information on Yelp consists of a blend of \textbf{list}, \textbf{grid}, and \textbf{gallery} layouts and other spatial arrangements.

In addition to discovery, Yelp supports various curation actions. Users can bookmark businesses they like within the system, thus performing \textbf{internal preservation of internal resources}. They can further augment information by either writing reviews or performing an \textbf{evaluation} of businesses or other reviews. Evaluation of businesses is enabled using a five-star rating system, and reviews can be evaluated by choosing between `Useful', `Funny', and `Cool' metrics. 

Identified gaps include lack of management mechanisms when businesses are bookmarked and lack of channel-picking mechanisms. A lot of information on Yelp is continuously updated, so channel-picking could help filter updated information. Furthermore, adding mechanisms for opportunistic navigation could make it possible to discover new restaurants every time and help the user when their information need is undefined.
 
} % end section

{\section{Discussion}
The evaluations of existing Web tools, provided in this chapter, using the conceptual framework for information discovery and curation demonstrate applicability of the framework. A set of questions, provided by the framework, can help in the process of tool evaluation, and they can also be applied in drawing distinctions between different tools. 

The framework associates different information discovery and curation actions with concrete mechanisms. However, it is not always clear which actions in the framework the tool needs to support. With help of the framework, some of the requirements (but not all) can be derived from the analysis of other applications, which might be in a similar domain or possess desired properties. 

Another source that can hint on which actions need support is the motive for discovery and curation activities in an application. For example, if the motive is to discover information with an undefined information need, the application  can either be tailored to support serendipitous discovery by providing opportunistic navigation mechanisms, or it can help the user to formulate the information need by suggesting search terms and categories.

Finally, the need for a given discovery or curation-supporting mechanism can be evaluated using intuition and experience of the designer. In some cases, it is especially difficult to estimate the importance of a mechanism in a specific application using known characteristics of the tool; however, as with many other decisions when it comes to developing or improving an application, the designer can use their judgment along with subsequent studies and evaluation. 

Evaluation and comparison of different Web applications can reveal useful insights about the mechanics of how the system induces user experience, and they can expose certain unadressed needs. The next Chapter illustrates the design process of a Web application that emerged from the evaluations of other tools using the conceptual framework. 

} % end section