\chapter{Methodology}
\label{chapter:methodology}

{\section{Building and Refining \\the Conceptual Framework}
\begin{table*}[htbp]
\small

\caption{Web-based Information Discovery and Curation Tools}

\begin{tabular}{|p{0.11\linewidth}| p{0.22\linewidth}| p{0.67\linewidth}|}

\hline
Application     & Description                                                                  & Summary of findings                                                                                                                                                                                                                                                                                            
\\
\hline
Pinterest       & \raggedright
Visual discovery tool, available at www.pinterest.com                        & - Supports serendipitous browsing, bookmark-based rediscovery, channel-based information discovery, and information curation. 

-Lacks support for search- and history-based rediscovery and fact finding.                                                                       \\
\hline
Delicious       & \raggedright
Social bookmarking service, available at delicious.com &                                                                - Supports channel-based discovery, bookmark-based rediscovery, and supports social curation. 

- Lacks support for visual link preview and list-based categorization. \\
\hline
Tumblr          & \raggedright Microblogging platform, available at www.tumblr.com                         & - Supports serendipitous browsing, bookmark-based rediscovery, channel-based information discovery.

- Lacks support for fact finding and list-based categorization.                                                                                                 \\
\hline
StumbleUpon     & \raggedright Web page discovery tool, available at www.stumbleupon.com                    & - Supports serendipitous browsing, bookmark- and history-based information rediscovery, channel-based information discovery, and information curation. 

- Lacks support for fact finding.                                                                       \\
\hline
Wikipedia       & \raggedright Free content Internet encyclopedia, available at en.wikipedia.org             & - Supports serendipitous discovery, fact finding, search-based rediscovery.

- Lacks support for history-based and bookmark-based rediscovery, personal preservation and resource evaluation. \\
\hline
Google Maps     & \raggedright Web mapping service, available at www.google.ca/maps                         & 
- Supports fact finding and rediscovery. 

- Lacks support for curation mechanisms, except for personal information preservation. 
\\
\hline
Rotten Tomatoes & \raggedright Movie and TV database, available at www.rottentomatoes.com                   & - Supports fact discovery, serendipitous browsing, and search-based rediscovery. 

-Lacks support for history-based and bookmark-based rediscovery, information preservation, and management. \\
\hline
500px           & \raggedright Photography site, available at 500px.com            & - Supports serendipitous browsing, channel-based discovery, and social curation. 

- Lacks support for fact discovery and list-based categorization. \\
\hline
BucketList      & \raggedright Goal tracking and discovery service, available at bucketlist.org             & - Supports serendipitous discovery, bookmark-based rediscovery, and channel-based discovery. 

- Lacks support for fact discovery, search- and history-based rediscovery. \\
\hline
We Heart It     & \raggedright Visual discovery tool, available at weheartit.com                            & - Supports serendipitous browsing, bookmark-based rediscovery, channel-based information discovery, and information curation.

- Lacks support for fact finding.                                                                       \\
\hline
Scoop.it!       & \raggedright Online publishing platform, available at www.scoop.it                        & - Supports serendipitous browsing, bookmark-based information rediscovery, channel-based information discovery, and information curation. 

- Lacks support for fact finding.                                                 \\
\hline
Google Images   & \raggedright Image discovery service, available at images.google.com                      & - Supports serendipitous browsing. 

- Lacks support for rediscovery, channel-based discovery, fact finding, or  information curation.                                                                                                         \\
\hline
Vimeo           & \raggedright Video sharing Website, available at vimeo.com                                & - Supports serendipitous discovery, bookmark-based rediscovery, and channel-based discovery, and information curation. 

- Lacks support for fact discovery and list-based categorization. \\
\hline
LifeHacker      & \raggedright Daily Weblog, available at lifehacker.com                                    & - Supports serendipitous discovery. 

- Lacks support for channel-based discovery and information curation.                                                                                                                                                                                                 \\
\hline
YouTube         & \raggedright Video hosting platform, available at www.youtube.com                         & - Allows for serendipitous discovery, channel-based discovery, history- and bookmark-based revisitation, and information curation. 

- Lacks support for internal sharing.                                                                                                                                                \\
\hline
Yelp            & \raggedright Business review site, available at www.yelp.ca                               & - Supports fact finding, serendipitous browsing, search-based rediscovery, certain aspects of information curation (e.g., evaluation and annotation).                                                                                                

 - Lacks support for channel-based discovery.   \\
\hline
IMDb            & \raggedright Movie database, available at www.imdb.com                                    & - Supports fact discovery, serendipitous discovery, and rediscovery. 

- Lacks support for channel-based discovery.                                                                                                                                                          \\
\hline
Trip Adviser    & \raggedright Travel site, available at www.tripadvisor.ca                                 & - Supports serendipitous discovery, fact finding, and personal information curation. 

- Lacks support for history-based rediscovery.                                                                                                                                 \\
\hline
Urban Spoon     & \raggedright Online bar and restaurant guide, available at www.urbanspoon.com             & 
- Supports serendipitous browsing, fact finding, evaluation and annotations. 

- Lacks support for channel-based discovery.  \\
\hline
Thesaurus       & \raggedright Online thesaurus, available at thesaurus.com                                 & - Supports serendipitous browsing and fact discovery. 

- Lacks support for information curation.                                                                                \\
\hline
\end{tabular}
\end{table*}
Development of our framework began with an extensive literature review. Although the previous section outlines only the key research that was considered, it illustrates the diversity of topics that contributed to forming an understanding of information seeking. From this review, we derived preliminary design factors. 

Through a careful analysis of 20 information discovery applications (see Table 1), we iteratively expanded the framework, added concepts, and established relations between those concepts. The framework can be expanded further, however, we selected the most popular information discovery applications in use today and considered the full range of features in those tools (both by referring to the literature and documentation on those tools, as well as exploring the features). The popularity of information discovery applications was determined using Website popularity ranks provided by Alexa\footnote[1]{Alexa is available at www.alexa.com}, a commercial Web traffic data provider. The focus was on applications that had strong information discovery components and lesser priority was given to applications whose purpose revolved only around curation. The framework was refined iteratively as we explored the literature and available tools, and for presentation purposes, we present the final version of the framework.

The exploration of information discovery tools was motivated by the following research questions:
\\

\emph{RQ1: How do existing Web applications support information discovery?}

\emph{RQ2: How do existing information discovery applications support information curation?}\\


We used Yin’s strategies for designing a case study ~\cite{yin} for guidance. The motivation behind choosing a case study over other methods of qualitative research was based on our choice of research questions (which have an explanatory nature), the lack of control over existing applications and their development, and having to focus on contemporary use of real-life Web applications. According to Yin ~\cite{yin}, a case study would be an optimal research strategy given the above characteristics.

For each case of our study, we chose a Web application whose primary purpose is to support information discovery. We examined the overall purpose of each application, its description as defined within the application, and literature and documentation related to the application (if they were available) against the features that the application provided. For example, if an application provided bookmarking features, we checked if it was indeed intended to be used for information preservation. 

To increase external validity of our study, we chose cases based on replication logic ~\cite{yin}. Using replication logic in case study design means carefully selecting each case so that it either predicts analogous results or predicts contrasting results but for anticipated reasons. Therefore, we used our preliminary conceptual framework to predict if an application supported each of the information discovery and curation tasks based on the features that the application provided. If our predictions were inaccurate, we would modify the framework accordingly and move onto the next case. 

Consequently, our methodology was an iterative process of selecting cases, analyzing them, and determining whether they could be described and evaluated using our framework. If we found a key feature that could not be described, we adapted the framework according to the findings. We repeated the process of case selection and evaluation until the framework was usable for all cases. We then grouped the elements of the framework into categories, recording corresponding questions to ask in order to evaluate applications. 

A list of the tools that were used as cases as well as brief summaries of our findings for each tool are presented in Table 1. Summaries are limited and provide a general idea of the results of examining the tools using the framework. Other tools were considered throughout the study, however, only the 20 applications presented underwent systematic examination. The framework itself is covered in the next section and presented in Table 2. Limitations of our study are outlined in Section 7.

} % end section