\chapter{Methodology}
\label{chapter:methodology}

Methodology for the study presented in this thesis consists of five major phases. To gain deeper understanding of the problem of information discovery and curation, (1) a systematic literature review was conducted. Based on the literature review, (2) I derived the preliminary information discovery and curation design factors and related them within a framework. The framework was then applied for evaluation of 20 different information discovery applications and iteratively refined after every evaluation. The resulting framework was applied to develop a place photo discovery application. Applying the framework for an application design revealed unaddressed gaps that were consequently. Lastly, the framework was applied to reevaluate some of the tools to validate its effectiveness.  

{\section{Research Questions}
The exploration of information discovery tools was motivated by the following research questions:

\emph{RQ1: How do existing Web applications support information discovery?}

\emph{RQ2: How do existing information discovery applications support information curation?}
}% end section

{\section{Literature Review}
Development of our framework began with an extensive literature review. Although the previous section outlines only the key research that was considered, it illustrates the diversity of topics that contributed to forming an understanding of information seeking. From this review, we derived preliminary design factors. Key findings are presented in Chapter~\ref{chapter:chapter_related_work}.
}% end section

{\section{Building and Refining \\the Conceptual Framework}
Through a careful analysis of 20 information discovery applications (see Table~\ref{table:tools}), I iteratively expanded the framework, added concepts, and established relations between those concepts. The framework can be expanded further, however, we selected the most popular information discovery applications in use today and considered the full range of features in those tools (both by referring to the literature and documentation on those tools, as well as exploring the features). The popularity of information discovery applications was determined using Website popularity ranks provided by Alexa\footnote[1]{Alexa is available at www.alexa.com}, a commercial Web traffic data provider. The focus was on applications that had strong information discovery components and lesser priority was given to applications whose purpose revolved only around curation. The framework was refined iteratively as we explored the literature and available tools, and for presentation purposes, we present the final version of the framework.

We used Yin’s strategies for designing a case study ~\cite{yin} for guidance. The motivation behind choosing a case study over other methods of qualitative research was based on our choice of research questions (which have an explanatory nature), the lack of control over existing applications and their development, and having to focus on contemporary use of real-life Web applications. According to Yin ~\cite{yin}, a case study would be an optimal research strategy given the above characteristics.

For each case of our study, we chose a Web application whose primary purpose is to support information discovery. We examined the overall purpose of each application, its description as defined within the application, and literature and documentation related to the application (if they were available) against the features that the application provided. For example, if an application provided bookmarking features, we checked if it was indeed intended to be used for information preservation. 

To increase external validity of our study, we chose cases based on replication logic ~\cite{yin}. Using replication logic in case study design means carefully selecting each case so that it either predicts analogous results or predicts contrasting results but for anticipated reasons. Therefore, we used our preliminary conceptual framework to predict if an application supported each of the information discovery and curation tasks based on the features that the application provided. If our predictions were inaccurate, we would modify the framework accordingly and move onto the next case. 

Consequently, our methodology was an iterative process of selecting cases, analyzing them, and determining whether they could be described and evaluated using our framework. If we found a key feature that could not be described, we adapted the framework according to the findings. We repeated the process of case selection and evaluation until the framework was usable for all cases. We then grouped the elements of the framework into categories, recording corresponding questions to ask in order to evaluate applications. 

A list of the tools that were used as cases as well as brief summaries of our findings for each tool are presented in Table 1. Summaries are limited and provide a general idea of the results of examining the tools using the framework. Other tools were considered throughout the study, however, only the 20 applications presented underwent systematic examination. The framework itself is covered in the next section and presented in Table~\ref{} 

\begin{table*}[htbp]
\small
\label{table:tools}
\caption{Web-based Information Discovery and Curation Tools}

\begin{tabular}{|p{0.20\linewidth}| p{0.30\linewidth}| p{0.45\linewidth}|}

\hline
Application     & Address                                                                  & Description                                                                                                                                                                                                                                                                                            
\\
\hline
Pinterest       & www.pinterest.com & Visual discovery tool \\
\hline
Delicious       & delicious.com & Social bookmarking service \\
\hline
Tumblr          & www.tumblr.com & Microblogging platform \\
\hline
StumbleUpon     & www.stumbleupon.com  & Web page discovery tool \\
\hline
Wikipedia       & en.wikipedia.org   & Free content Internet encyclopedia\\
\hline
Google Maps     & www.google.ca/maps  & Web mapping service\\
\hline
Rotten Tomatoes & www.rottentomatoes.com & Movie and TV database\\
\hline
500px           & 500px.com            & Photography site\\
\hline
BucketList      & bucketlist.org  & Goal tracking and discovery service\\
\hline
We Heart It     & weheartit.com & Visual discovery tool \\
\hline
Scoop.it!       & www.scoop.it & Online publishing platform \\
\hline
Google Images   & images.google.com  &Image discovery service \\
\hline
Vimeo           & vimeo.com  & Video sharing Website\\
\hline
LifeHacker      & lifehacker.com                                    & Daily Weblog \\
\hline
YouTube         & www.youtube.com 	& Video hosting platform \\
\hline
Yelp            & www.yelp.ca  & Business review site\\
\hline
IMDb            & www.imdb.com  & Movie database \\
\hline
Trip Adviser    & www.tripadvisor.ca & Travel site \\
\hline
Urban Spoon     & www.urbanspoon.com             & Online bar and restaurant guide\\
\hline
Thesaurus       & thesaurus.com                                 & Online thesaurus \\
\hline
\end{tabular}
\end{table*}

} % end section

{\section{Developing Application}

}% end section

{\section{Revising and Extending \\the Conceptual Framework}

}% end section

{\section{Reevaluation of the Framework}
In order to finalize the framework, it was applied to reevaluate five of the previously examined tools. 
}% end section

{\section{Limitations}
The case study we conducted has a number of limitations: a lack of documentation, literature, and formal descriptions of available features for some applications introduces a threat to construct validity of the study. In addition, information discovery tools and features can be used in manners unintended or unforeseen by designers and developers. Therefore, the use of some features within information discovery applications was recorded based on the researchers' interpretations. To compensate for such limitations, the researchers employed the tools for personal use over an extended period of time to gain a deeper understanding of their use. In addition, the researchers considered some cases with repeating functionality and design to be able to validate or clarify prior findings. 

Many Web applications rapidly evolve. Therefore, our tool analysis only applies to tools at the moment of our study.

Only Web applications running in browsers on a desktop computer were considered in this study. Our study can be extended with use of various devices, such as smartphones and tablets, as information discovery patterns and mechanisms may very for different platforms. 

Another limitation was the lack of prior research studies on the subject matter. Some researchers have studied information seeking models and high-level Web tasks, but there is a lack of literature on how to enable and support different Web tasks. This opens up opportunities for future research to analyze methods of developing and building frameworks for facilitating and evaluating tools that support other Web tasks, such as communication, transactions, and goal realization.


} % end section

