\chapter{A Preliminary Framework for Information Discovery and Curation}
\label{chapter:old_framework}
A preliminary framework for information discovery and curation (see Tables~\ref{table:old_framework_discovery} and~\ref{table:old_framework_curation}) was designed in hopes to merge the gap between existing high-level information behaviour models and existing Web tools~\cite{voyloshnikova2014}. It was constructed by identifying important elements in current Web applications (see Table~\ref{table:tools}) and relating them among themselves with help of existing literature highlighted in Chapter~\ref{chapter:chapter_related_work}. The framework in this chapter only represents the preliminary efforts, and it is presented with the goal to illustrate how the framework evolved and outline some of the challenges with its construction.  

{\section{The Preliminary Framework Composition}
The framework consists out of two main categories of design factors, discovery and curation, that are consequently bro
\textit{Serendipitous discovery} refers to information discovery resulting from serendipitous browsing. Such discovery is characterized by under-defined, absent, or hidden information needs, and it usually involves browsing through diverse resources with varying content types~\cite{kellar2006goal, kellar2007field}. Here, resource is defined as a collection of information about a single unit of inquiry, usually bundled together for presentation purposes. Some examples of resources are places, images, blog posts, and Web pages.

\textit{Fact discovery} refers to information discovery resulting from the search for a specific piece of information. It is characterized by a well-defined information need and is easier to perform within systems that provide access to homogeneous types of information~\cite{kellar2006goal, lindley2012s}.

\textit{Rediscovery} refers to information discovery resulting from revisiting previously discovered resources~\cite{tauscher1997people}. Rediscovery can be enabled using search, history, or bookmarking.

\textit{Channel-based discovery} can incorporate two different information seeking tasks, monitoring and awareness. It occurs when information is suggested to users based on the content that they are subscribed to. If users can actively look for updates, then an application affords monitoring~\cite{morrison2001taxonomic}. If users can receive notifications about updates, then an application facilitates awareness~\cite{bates1986exploratory,bates2002toward}. Channel-based information discovery is usually enabled at sites that have regularly updated content, such as Pinterest and YouTube.    

\textit{Management} of information can be performed through organizing information into lists (or collections) or tagging privately (for personal use) or publicly.  

To \textit{preserve} information, people use diverse bookmarking mechanisms. Information can be preserved within the same application where it was found and in a different application.

} % end section
{\section{Challenges and Issues with the Construction of the Preliminary Framework}
Although the preliminary framework was good at breaking down the tasks that people perform when discovering information, some of the aspects require clarification. 
} % end section


\begin{table*}[htbp]
\caption{Preliminary Framework - Discovery}
\centering
\label{table:old_framework_discovery}
\footnotesize
\begin{tabular}{|p{0.31\linewidth}|p{0.64\linewidth}|}
\hline
\textbf{\small{Design factors}}   & \textbf{\small{Questions to be posed during the design or evaluation of Web-based information discovery tools 
}}  \\
\hline
\emph{\textbf{Serendipitous discovery}}     &                                                                                                           \\

Arbitrary navigation         & Does the application provide a means for arbitrary navigation among resources?                              \\
Search-based navigation      & Does the search engine help retrieve diverse resources related to the topic of interest?               \\
Category-guided navigation & Do categories suggest and help with navigating to resources related to the topic of interest?           \\
Integration                  & If resources originate from a different site, do they link to their original sources?                   \\
Visual link preview               & If resources are delivered as links, do they have visual previews?                                                                        \\
Spatial arrangement          & Is there a semantic to the spatial arrangement of resources?                                                    \\

\emph{\textbf{Fact discovery}}                &                                                                                                           \\
Search-based navigation      & Does the search feature help discover the specific resource of interest?                                  \\
Category-guided navigation & Do categories help narrow results to specific types of resources?                                   \\
Integration                  & If resources originate from a different site, do they link to their original sources?                   \\
Uniform representation       & If resources are uniform, are they presented in a uniform way? \\
Visual link preview               & If resources are delivered as links, do they have visual previews?                                                                        \\
Spatial arrangement          & Is there a semantic to the spatial arrangement of resources?                                                    \\

\emph{\textbf{Rediscovery}}                     &                                                                                                           \\
History-based rediscovery    & Does the application save and provide access to browsing history?                                        \\
Bookmark-based rediscovery   & Does the application support bookmark-based resource revisitation?                                        \\
Search-based rediscovery     & Is the search a reliable method for resource revisitation?                             \\

\emph{\textbf{Channel-based discovery}}          &                                                                                                           \\
Site subscription            & Does the application allow subscriptions to news and updates?                                             \\
User subscription             & Does the application allow subscriptions to other users' activities?                                      \\
Notifications                & Does the application have one or more notification mechanisms?                                                      \\
Subscription to news feed                  & Can subscription updates be visible within the application?  \\
Content news feed                  & Can content updates be visible within the application? \\

\hline     

         
\end{tabular}
\end{table*}


\begin{table*}[htbp]
\caption{Preliminary Framework - Curation}
\centering
\label{table:old_framework_curation}
\footnotesize
\begin{tabular}{|p{0.25\linewidth}|p{0.65\linewidth}|}
\hline
\textbf{\small{Design factors}}   & \textbf{\small{Questions to be posed during the design or evaluation of Web-based information discovery tools 
}}  \\

\hline          
\emph{\textbf{Management}}                    &                                                                                                           \\
List-based categorization               & Does the application support sorting information into list-like structures, either privately or publicly?                                                  \\
Tag-based categorization               & Does the application support tagging, either privately or publicly?                                                  \\

\emph{\textbf{Preservation}}                   &                                                                                                           \\
Internal preservation of internal resources       & Does the application support bookmarking mechanism(s) for preserving internal information within the application?        \\
Internal preservation of external resources       & Does the application support bookmarking mechanism(s) for preserving external information within the application?        \\
External preservation of internal resources      & Does the application support bookmarking mechanism(s) for preserving internal information outside of the application? \\ 

\emph{\textbf{Augmentation}}            &                                                                                                           \\
Evaluation                   & Can the resource evaluations be recorded privately or publicly? \\
Annotation                   & Can resources be annotated privately or publicly?                                                                               \\    
       
\emph{\textbf{Sharing}}            &                                                                                                           \\
Adding resources             & Can resources be publicly added to the collection of information within the application from other Web pages?     \\
Internal sharing         & Can internal resources be publicly reshared within the application?         \\ 
External sharing          & Can internal resources be publicly reshared outside of the application?         \\ 
         
\hline
\end{tabular}
\end{table*}





