\chapter{A Preliminary Framework for Information Discovery and Curation}
\label{chapter:old_framework}
A preliminary framework for information discovery and curation (see Tables~\ref{table:old_framework_discovery} and~\ref{table:old_framework_curation}) was designed in hopes of merging the gap between existing Web tools and high-level information behaviour models~\cite{voyloshnikova2014}. It was constructed by identifying important elements in current Web applications (see Table~\ref{table:tools}) and relating them among themselves with the help of background research (see Chapter~\ref{chapter:chapter_related_work}). In this chapter, I describe the preliminary version of the framework to illustrate its evolution and outline some of the challenges with its construction. The final framework is discussed in Chapter~\ref{chapter:framework}.

{\section{Preliminary Framework Composition}
The two main parts of the framework (discovery and curation) encapsulate other categories of design factors for Web applications. Serendipitous discovery, fact discovery, rediscovery, and channel-based discovery are the main types of information discovery. Curation consists of common curation tasks: information management, preservation, augmentation, and sharing. This section provides brief summaries of each part of the framework.

\textit{Serendipitous discovery} refers to information discovery resulting from serendipitous browsing. Such discovery is characterized by underdefined, absent, or hidden information needs, and it usually involves browsing through diverse resources with varying content types~\cite{kellar2006goal, kellar2007field}. Here, a resource is defined as a collection of information about a single unit of inquiry, usually bundled together for presentation purposes. Some examples of resources are places, images, blog posts, and Web pages. Serendipitous discovery can be supported using arbitrary, search-based, and category-based navigation mechanisms, integration, visual link preview, and spatial arrangement of resources.

\textit{Fact discovery} is defined as information discovery resulting from the search for a specific piece of information. It is characterized by a well-defined information need and is easier to perform within systems that provide access to homogeneous types of information~\cite{kellar2006goal, lindley2012s}. Fact discovery can be supported using category-based and search-based navigation mechanisms, integration, uniform representation, visual link preview, and spatial arrangement of resources, as with serendipitous discovery. 

\textit{Rediscovery} refers to information discovery resulting from revisiting previously discovered resources~\cite{tauscher1997people}. Rediscovery can be enabled using search, history, or bookmarking.

\textit{Channel-based discovery} can incorporate two different information seeking tasks, monitoring and awareness. It occurs when information is suggested to users based on the content they are subscribed to. If users can actively look for updates, then an application affords monitoring~\cite{morrison2001taxonomic}. If users can receive notifications about updates, then an application facilitates awareness~\cite{bates1986exploratory,bates2002toward}. Channel-based information discovery is usually enabled on sites that have regularly updated content, such as Pinterest and YouTube. Channel-based discovery can be supported using site, user, and news feed subscriptions, notifications, and by displaying the news feed.   

\textit{Management} of information can be performed through organizing information into lists (or collections) or tagging publicly or privately.  

To \textit{preserve} information, people use diverse bookmarking mechanisms. Information can be preserved within the application where it was found or in a different application. As another form of preservation, internal preservation of external resources, new information can be added to the Web application in question.

Information \textit{augmentation} is the notion of adding value to existing digital assets~\cite{beagrie2008digital}. Augmentation can be accomplished through activities such as rating, commenting, describing, and upvoting. In other words, by augmenting or evaluating information.  

Information \textit{sharing} is commonly performed within information discovery and curation applications. It is a way to communicate information to other individuals or groups of people though various Web channels. Information communication is an important aspect of Wilson's information behaviour model~\cite{wilson1981user}. In information discovery and curation tools, information sharing can be enabled by providing mechanisms for publicly adding resources, resharing resources within the same application or outside of it, in other applications.

The preliminary framework aims to help with the evaluation and design of currently existing tools, however, it has certain shortcomings, which are outlined in the next section.

} % end section
{\section{Limitations of the Preliminary Framework}
Although the preliminary framework can be applied to evaluate some aspects of information discovery and curation for Web applications, some of its characteristics make it difficult to use.

In the preliminary framework, there is a clear distinction between the types of curation and discovery subcategories. Discovery subcategories represent \textbf{types} of information discovery (serendipitous discovery, fact discovery, etc), whereas curation subcategories represent curation \textbf{tasks}. For comparison, information discovery tasks can include \textit{navigating} to a target resource or \textit{exploring} a resource in order to extract necessary information, whereas curation tasks would be to \textit{preserve} a resource or to \textit{manage} a collection of resources.

Furthermore, the types of information discovery in the framework are mutually independent. Serendipitous and fact discoveries are defined using specificity of the user's information need. Defined information needs result in fact discovery, and undefined information needs result in serendipitous discovery. However, rediscovery and channel-based discovery are defined mostly by how the information in question is related to the user, whether or not it has been discovered before, or if the user chose to receive it. Therefore, there can be serendipitous rediscovery, channel-based fact discovery, etc. 

\pagebreak

Another aspect of information discovery and curation support that the framework fails to thoroughly address is the ways in which the system provides cognitive support to the user and reduces the amount of effort the user needs to put in to perform a task. Examples of such support are automatic sharing of curated content and suggestion of search terms to the user. The framework has to be extended beyond just the factors that \textbf{enable} information discovery and curation and showcase strategies that can help \textbf{improve} these enabling mechanisms.

In the next chapter, I present the final version of the framework that addresses some of the major drawbacks of the preliminary framework.  
} % end section


\begin{table*}[ht!]
\caption{Preliminary Framework - Discovery}
\centering
\label{table:old_framework_discovery}
\footnotesize
\begin{tabular}{|p{0.31\linewidth}|p{0.64\linewidth}|}
\hline
\textbf{\small{Design factors}}   & \textbf{\small{Questions to be posed during the design or evaluation of Web-based information discovery tools 
}}  \\
\hline
\emph{\textbf{Serendipitous discovery}}     &                                                                                                           \\

Arbitrary navigation         & Does the application provide a means for arbitrary navigation among resources?                              \\
Search-based navigation      & Does the search engine help retrieve diverse resources related to the topic of interest?               \\
Category-guided navigation & Do categories suggest and help with navigating to resources related to the topic of interest?           \\
Visual link preview               & If resources are delivered as links, do they have visual previews?                                                                        \\
Spatial arrangement          & Is there a semantic to the spatial arrangement of resources?                                                  \\
Integration                  & If resources originate from a different site, do they link to their original sources?                   \\  

\emph{\textbf{Fact discovery}}                &                                                                                                           \\
Search-based navigation      & Does the search feature help discover the specific resource of interest?                                  \\
Category-guided navigation & Do categories help narrow results to specific types of resources?                                   \\
Uniform representation       & If resources are uniform, are they presented in a uniform way? \\
Visual link preview               & If resources are delivered as links, do they have visual previews?                                                                        \\
Spatial arrangement          & Is there a semantic to the spatial arrangement of resources?                                                    \\
Integration                  & If resources originate from a different site, do they link to their original sources?                   \\

\emph{\textbf{Rediscovery}}                     &                                                                                                           \\
Search-based rediscovery     & Is the search a reliable method for resource revisitation?                             \\
History-based rediscovery    & Does the application save and provide access to browsing history?                                        \\
Bookmark-based rediscovery   & Does the application support bookmark-based resource revisitation?                                        \\


\emph{\textbf{Channel-based discovery}}          &                                                                                                           \\
Site subscription            & Does the application allow subscriptions to news and updates?                                             \\
User subscription             & Does the application allow subscriptions to other users' activities?                                      \\
Notifications                & Does the application have one or more notification mechanisms?                                                      \\
Subscription to news feed                  & Are subscription updates visible within the application?  \\
Content news feed                  & Are content updates visible within the application? \\

\hline     

         
\end{tabular}
\end{table*}


\begin{table*}[htbp]
\caption{Preliminary Framework - Curation}
\centering
\label{table:old_framework_curation}
\footnotesize
\begin{tabular}{|p{0.25\linewidth}|p{0.65\linewidth}|}
\hline
\textbf{\small{Design factors}}   & \textbf{\small{Questions to be posed during the design or evaluation of Web-based information discovery tools 
}}  \\

\hline          
\emph{\textbf{Management}}                    &                                                                                                           \\
List-based categorization               & Does the application support sorting information into list-like structures, either privately or publicly?                                                  \\
Tag-based categorization               & Does the application support tagging, either privately or publicly?                                                  \\

\emph{\textbf{Preservation}}                   &                                                                                                           \\
Internal preservation of internal resources       & Does the application support bookmarking mechanism(s) for preserving internal information within the application?        \\
Internal preservation of external resources       & Does the application support bookmarking mechanism(s) for preserving external information within the application?        \\
External preservation of internal resources      & Does the application support bookmarking mechanism(s) for preserving internal information outside of the application? \\ 
&\\
\emph{\textbf{Augmentation}}            &                                                                                                           \\
Evaluation                   & Can resource evaluations be recorded privately or publicly? \\
Annotation                   & Can resources be annotated privately or publicly?                                                                               \\    
 &\\      
\emph{\textbf{Sharing}}            &                                                                                                           \\
Adding resources             & Can resources be publicly added to the collection of information within the application from other Web pages?     \\
Internal sharing         & Can internal resources be publicly reshared within the application?         \\ 
External sharing          & Can internal resources be publicly reshared outside of the application?         \\ 
         
\hline
\end{tabular}
\end{table*}





