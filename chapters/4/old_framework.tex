\chapter{A Preliminary Framework for Information Discovery and Curation}
\label{chapter:old_framework}
A preliminary framework for information discovery and curation (see Tables~\ref{table:old_framework_discovery} and~\ref{table:old_framework_curation}) was designed in hopes to merge the gap between existing high-level information behaviour models and existing Web tools~\cite{voyloshnikova2014}. It was constructed by identifying important elements in current Web applications (see Table~\ref{table:tools}) and relating them among themselves with help of background research highlighted in Chapter~\ref{chapter:chapter_related_work}. The framework in this chapter only represents the preliminary efforts, and it is presented with the goal to illustrate how the framework evolved as well as outline some of the challenges with its construction.  

{\section{Preliminary Framework Composition}
The two main parts of the framework (discovery and curation) encapsulate other categories of design factors for Web applications. Serendipitous discovery, fact discovery, rediscovery, and channel-based discovery are the main types of information discovery. Curation consists of four most commonly performed curation tasks: information management, preservation, augmentation, and sharing. This section provides brief summaries of each part of the framework.

\textit{Serendipitous discovery} refers to information discovery resulting from serendipitous browsing. Such discovery is characterized by underdefined, absent, or hidden information needs, and it usually involves browsing through diverse resources with varying content types~\cite{kellar2006goal, kellar2007field}. Here, resource is defined as a collection of information about a single unit of inquiry, usually bundled together for presentation purposes. Some examples of resources are places, images, blog posts, and Web pages. 

\textit{Fact discovery} is defined as information discovery resulting from the search for a specific piece of information. It is characterized by a well-defined information need and is easier to perform within systems that provide access to homogeneous types of information~\cite{kellar2006goal, lindley2012s}.

\textit{Rediscovery} refers to information discovery resulting from revisiting previously discovered resources~\cite{tauscher1997people}. Rediscovery can be enabled using search, history, or bookmarking.

\textit{Channel-based discovery} can incorporate two different information seeking tasks, monitoring and awareness. It occurs when information is suggested to users based on the content that they are subscribed to. If users can actively look for updates, then an application affords monitoring~\cite{morrison2001taxonomic}. If users can receive notifications about updates, then an application facilitates awareness~\cite{bates1986exploratory,bates2002toward}. Channel-based information discovery is usually enabled at sites that have regularly updated content, such as Pinterest and YouTube.    

\textit{Management} of information can be performed through organizing information into lists (or collections) or tagging publicly or privately, for personal use.  

To \textit{preserve} information, people use diverse bookmarking mechanisms. Information can be preserved within the same application where it was found or in a different application. As another form of preservation, new information can be added to the Web application in question.

Information \textit{augmentation} is the notion of adding value to existing digital assets~\citep{beagrie2008digital}. Augmentation can be accomplished through activities such as rating, commenting, describing, upvoting, etc.  

Information \textit{sharing} is commonly performed within information discovery and curation applications. It is a way to communicate information to other individuals or groups of people though various Web channels. Information communication is an important aspect of Wilson's information behaviour model~\cite{wilson1981user}.

The preliminary framework aims to help with evaluation and design of currently existing tools; however, it is built not without certain shortcomings, which are outlined in the next section.

} % end section
{\section{Limitations of the Preliminary Framework}
Although the preliminary framework can be applied to evaluate some aspects of information discovery and curation for Web applications, some of characteristics of the framework make it difficult to use.

There is a clear distinction between the types of curation and discovery subcategories. Discovery subcategories represent \textbf{types} of information discovery (serendipitous discovery, fact discovery, etc) whereas curation subcategories represent curation \textbf{tasks}. For comparison, information discovery tasks would be \textit{navigating} to a target resource or \textit{exploring} a resource in order to extract necessary information.

Furthermore, the types of information discovery in the framework are not parallel to each other. Serendipitous and fact discoveries are defined using specificity of the user's information need. Defined information need results in fact discovery and undefined in serendipitous discovery. However, rediscovery and channel-based discovery are defined mostly by how the information in question is related to the user, whether or not it has been discovered before or if the user chose to receive it. Therefore, there can be serendipitous rediscovery, channel-based fact discovery, etc. 

Another aspect of information discovery and curation support that the framework fails to thoroughly address is the ways in which the system provides cognitive support to the user and reduces the amount of effort the user needs to put is in order to perform a task. Examples of such support are automatic sharing of curated content and suggesting search terms to the user. The framework has to be extended beyond the factors that \textbf{enable} information discovery and curation and showcase strategies that can help \textbf{improve} mechanisms that enable given factors.

Addressing some of the major drawbacks of the preliminary framework, Chapter~\ref{chapter:framework} presents the final version of the framework. 
} % end section


\begin{table*}[htbp]
\caption{Preliminary Framework - Discovery}
\centering
\label{table:old_framework_discovery}
\footnotesize
\begin{tabular}{|p{0.31\linewidth}|p{0.64\linewidth}|}
\hline
\textbf{\small{Design factors}}   & \textbf{\small{Questions to be posed during the design or evaluation of Web-based information discovery tools 
}}  \\
\hline
\emph{\textbf{Serendipitous discovery}}     &                                                                                                           \\

Arbitrary navigation         & Does the application provide a means for arbitrary navigation among resources?                              \\
Search-based navigation      & Does the search engine help retrieve diverse resources related to the topic of interest?               \\
Category-guided navigation & Do categories suggest and help with navigating to resources related to the topic of interest?           \\
Integration                  & If resources originate from a different site, do they link to their original sources?                   \\
Visual link preview               & If resources are delivered as links, do they have visual previews?                                                                        \\
Spatial arrangement          & Is there a semantic to the spatial arrangement of resources?                                                    \\

\emph{\textbf{Fact discovery}}                &                                                                                                           \\
Search-based navigation      & Does the search feature help discover the specific resource of interest?                                  \\
Category-guided navigation & Do categories help narrow results to specific types of resources?                                   \\
Integration                  & If resources originate from a different site, do they link to their original sources?                   \\
Uniform representation       & If resources are uniform, are they presented in a uniform way? \\
Visual link preview               & If resources are delivered as links, do they have visual previews?                                                                        \\
Spatial arrangement          & Is there a semantic to the spatial arrangement of resources?                                                    \\

\emph{\textbf{Rediscovery}}                     &                                                                                                           \\
History-based rediscovery    & Does the application save and provide access to browsing history?                                        \\
Bookmark-based rediscovery   & Does the application support bookmark-based resource revisitation?                                        \\
Search-based rediscovery     & Is the search a reliable method for resource revisitation?                             \\

\emph{\textbf{Channel-based discovery}}          &                                                                                                           \\
Site subscription            & Does the application allow subscriptions to news and updates?                                             \\
User subscription             & Does the application allow subscriptions to other users' activities?                                      \\
Notifications                & Does the application have one or more notification mechanisms?                                                      \\
Subscription to news feed                  & Can subscription updates be visible within the application?  \\
Content news feed                  & Can content updates be visible within the application? \\

\hline     

         
\end{tabular}
\end{table*}


\begin{table*}[htbp]
\caption{Preliminary Framework - Curation}
\centering
\label{table:old_framework_curation}
\footnotesize
\begin{tabular}{|p{0.25\linewidth}|p{0.65\linewidth}|}
\hline
\textbf{\small{Design factors}}   & \textbf{\small{Questions to be posed during the design or evaluation of Web-based information discovery tools 
}}  \\

\hline          
\emph{\textbf{Management}}                    &                                                                                                           \\
List-based categorization               & Does the application support sorting information into list-like structures, either privately or publicly?                                                  \\
Tag-based categorization               & Does the application support tagging, either privately or publicly?                                                  \\

\emph{\textbf{Preservation}}                   &                                                                                                           \\
Internal preservation of internal resources       & Does the application support bookmarking mechanism(s) for preserving internal information within the application?        \\
Internal preservation of external resources       & Does the application support bookmarking mechanism(s) for preserving external information within the application?        \\
External preservation of internal resources      & Does the application support bookmarking mechanism(s) for preserving internal information outside of the application? \\ 

\emph{\textbf{Augmentation}}            &                                                                                                           \\
Evaluation                   & Can the resource evaluations be recorded privately or publicly? \\
Annotation                   & Can resources be annotated privately or publicly?                                                                               \\    
       
\emph{\textbf{Sharing}}            &                                                                                                           \\
Adding resources             & Can resources be publicly added to the collection of information within the application from other Web pages?     \\
Internal sharing         & Can internal resources be publicly reshared within the application?         \\ 
External sharing          & Can internal resources be publicly reshared outside of the application?         \\ 
         
\hline
\end{tabular}
\end{table*}





