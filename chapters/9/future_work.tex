\chapter{Future Work and Conclusions}
\label{chapter:future_work}


In my study, I analyzed information curation and seeking tasks and developed a conceptual framework of factors and questions that are important when building and evaluating Web information discovery and curation tools. I then evaluated and iteratively refined the framework by analyzing twenty different information discovery applications and provided concrete examples of tool support addressing various concepts of the framework. Finally, I designed a Web-based application for place photo discovery and curation using the conceptual framework, and validated the framework by reevaluating five of the previously examined tools.

The current version of the framework is generalized to be applicable to most information discovery applications. Finding ways to instantiate the framework and extend it to be used in domain-specific practices could serve as a potential research goal in the future. For example, video discovery and curation activities have unique properties related to the type of data to be discovered --- information is mostly found in the video itself, and it cannot be viewed all at the same time. Hence, the framework could be extended to address domain-specific challenges. 

Another potential research direction would be to expand my investigation to include factors that influence the need for one information discovery type over another and further deepen an understanding of the relationship between the motives for information discovery and curation activities and information discovery types. 

\pagebreak
Additionally, one could investigate how collaboration in information discovery and curation relates to the conceptual framework. Generally, collaboration mechanisms in most Web information discovery applications are limited to information sharing, public information augmentation and tagging. However, collaboration often involves other activities, such as communication, coordination, and other domain-specific shared activities.

My framework opens up opportunities for structured information discovery and curation tool evaluation and design. As more tools are being developed within the social space of information discovery and curation, understanding how these tasks can be supported promises advancements in how Web applications are designed.




