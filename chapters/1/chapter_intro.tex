\chapter{Introduction}
\label{chapter:chapter_intro}

Today, people use Web technologies to satisfy their information needs. People research their interests and hobbies using various online resources, shoppers search online stores for product characteristics to make purchasing decisions, and travelers visit online booking sites to find information about flights and hotels. In order to accommodate diverse and evolving user needs, Web applications continuously introduce new features and services, empowering information discovery and curation. 


The term ``information discovery'' has been used by many researchers to define or explain various information behaviour paradigms, such as information exploration~\cite{waterworth1991model}, serendipitous information seeking~\cite{foster2003serendipity}, etc. However, the definition of information discovery itself is difficult to pinpoint. 

Proper and Bruza (1999) use the phrase information discovery to apply solely to the identification and retrieval of relevant content from electronic sources. One of the first researchers on the topic of information discovery, Bruner~\cite{bruner1961act}, defined information discovery as ``all forms of obtaining knowledge for oneself by the use of one's own mind''. 


I build on Bruner’s definition of information discovery to underline the fact t, 


Web information discovery refers to fining information online~\cite{kamran2011discovery}. Sometimes, Web users hope to find particular pieces of information, such as show times and phone numbers, to satisfy specific information needs ~\cite{proper}. Other times, users lack well-articulated information needs so they engage in opportunistic browsing ~\cite{lindley}. Often, people discover information online without even looking for it ~\cite{bates1986}. The nature of information discovery can vary, and therefore, it requires elaborate tool support. Required functionality for information discovery and curation can also be distributed among two or more applications, which often leads to tools providing integrated solutions.

In addition, people perform information curation tasks, such as management and preservation, to maintain and add value to collections of information ~\cite{beagrie}. With the rapidly increasing popularity of socially-curated information spaces, it is important to understand how to enable management and curation activities when designing tools that support information discovery.

To close their knowledge gaps, people turn to various Web technologies ranging from specialized search tools to visual discovery applications. Several studies have been directed at exploring high-level Web tasks, including information seeking tasks ~\cite{kellar2006, kellar2007, morrison, sellen}, deriving models of information seeking behaviors ~\cite{choo, ellis1989, ellis1993, ellis1997, bates1986, bates2002}, and looking at methods of information curation ~\cite{beagrie, wittaker}. However, more research is necessary to determine how different tools and their features provide fundamental support for information discovery and curation.

To enhance information seeking and curating experiences and support users' interactions, I extend existing research by (1) deriving factors that enable information discovery and curation and relating them within a framework, (2) using the framework to establish a set of questions that can be used when evaluating and designing new applications, and (3) iteratively evaluating the framework by using it to study and describe current Web applications, which in turn helped us refine the framework of factors and questions. In summary, the framework addresses our research goal which is to gain an understanding of how existing tools support digital information curation and discovery. 



