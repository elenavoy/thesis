\chapter{Introduction}
\label{chapter:chapter_intro}

Today, people use Web technologies to satisfy their information needs. People research their interests and hobbies using various online resources, shoppers search online stores for product characteristics to make purchasing decisions, and travelers visit online booking sites to find information about flights and hotels. In order to accommodate diverse and evolving user needs, Web applications continuously introduce new features and services, empowering information discovery and curation. 


The term ``information discovery'' has been used by many researchers to define or explain various information behaviour paradigms, such as information exploration~\cite{waterworth1991model}, serendipitous information seeking~\cite{foster2003serendipity}, etc. However, the definition of information discovery itself is difficult to pinpoint. 

Lynch describes resource discovery as a complex collection of activities ranging from locating a well-specified information to iterative research activities, that in turn can involve identification
of potentially relevant resources, organization and ranking of resources, and resource exploration~\cite{lynch1995networked}. Proper and Bruza apply the term ``information discovery'' in the context of  the identification and retrieval of relevant information from electronic sources~\cite{proper1999information}. 

In the field of cognitive psychology, Jerome S. Bruner~\cite{bruner1961act} defines information discovery as ``all forms of obtaining knowledge for oneself by the use of one's own mind.'' I build on Bruners's definition to underline the importance of the cognitive processes that govern information discovery. Therefore, \textit{information discovery} is a process of obtaining knowledge from digital sources that can involve complex mental tasks and information behavior.  

Information behavior, which refers to the totality of ways in which humans interact with information~\cite{wilson2000human}, can enable and support information discovery when targeted at information maintenance and augmentation. This type of information behavior is also known as \textit{digital curation}.

Just as the term ``information discovery'', the term "digital curation" is perceived differently across various disciplines and among researchers. In this thesis, I use the definition proposed by Giaretta~\cite{giaretta2006dcc} and adopted by the Digital Curation Centre\footnote{The Digital Curation Centre is a UK-based organization established to support expertise and practice in digital curation and preservation across communities of practice.} which states that digital curation is a process of maintaining and adding value to existing body of information to improve its future use and retrieval.   

Information discovery can take on many forms. Web users might be hoping to find particular pieces of information, such as show times and phone numbers, to satisfy specific information needs~\cite{proper1999information}, or they might be lacking well-articulated information needs so they engage in opportunistic browsing~\cite{lindley2012s}. Sometimes, people discover information online without even looking for it~\cite{bates1986exploratory}. The nature of information discovery can vary, and therefore, it requires elaborate tool support. Required functionality for information discovery and curation can also be distributed among two or more applications, which often leads to tools providing integrated solutions. With people having such diverse information needs and methods of looking for information, designing for information discovery is a challenging task~\cite{conaway2010designing, marchionini2006exploratory}.

Addressing the problem of designing Web applications for information discovery, my research goal is to gain an understanding of how existing tools support digital information discovery and curation. While several researchers propose frameworks targeted at design of information discovery systems~\cite{proper1999information, kerne2004information}, the importance of information curation in realm of information discovery has been largely overlooked despite the rapidly increasing popularity of socially-curated information spaces. Moreover, many of existing studies that focus on how people look for and discover information online~\cite{bates1986exploratory, bates2002toward, choo2000information, ellis1989behavioural, ellis1993comparison, kellar2006goal, kellar2007field, lindley2012s, morrison2001taxonomic, sellen2002knowledge} fail to examine concrete features of existing Web-based information discovery applications, which empower real-world users. More research is necessary to determine how different tools and their features provide fundamental support for information discovery and curation.

To enhance information seeking and curating experiences and support users' interactions, I extend existing research by (1) deriving factors that enable information discovery and curation and relating them within a framework, (2) using the framework to establish a set of questions that can be used when evaluating and designing new applications, (3) iteratively evaluating the framework by using it to study and describe current Web applications and to design a new application, which in turn helped refine the framework of factors and questions, and (4) relating the framework to user information discovery goals, that drive the underlying usage of many Web-based applications. 

This thesis is organized as follows. The methodology and the process of building and refining a conceptual framework is documented in Chapter~\ref{chapter:methodology}. Chapter~\ref{chapter:chapter_related_work} highlights some of the studies and technologies related to information discovery and curation tasks. Chapter~\ref{chapter:old_framework} describes preliminary attempts of building the conceptual framework and outlines its shortcomings. In Chapter~\ref{chapter:goals}, I expand on the goals that motivate information discovery. Chapter~\ref{chapter:framework} outlines the conceptual framework and provides questions that enable digital information discovery and support curation, as well as gives specific examples from real-world Web applications. In Chapter~\ref{chapter:application}, I demonstrate how the framework can be used to reveal missing features in tools and propose new directions for development with relation to user goals. Chapter~\ref{chapter:implications} summarizes implications for research and practice, followed by future work and conclusions in Chapter~\ref{chapter:future_work}.




