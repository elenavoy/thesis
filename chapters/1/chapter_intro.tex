\chapter{Introduction}
\label{chapter:chapter_intro}

Web technologies help people satisfy their information needs. People research their interests and hobbies using various online resources, shoppers search online stores for product characteristics to make purchasing decisions, and travelers visit online booking sites to find information about flights and hotels. In order to accommodate diverse and evolving user needs, Web applications continuously introduce new features and services, empowering information discovery and curation. 

The term ``information discovery'' has been used by many researchers to define or explain various information behaviour paradigms, such as information exploration~\cite{waterworth1991model} and serendipitous information seeking~\cite{foster2003serendipity}. However, the definition of information discovery itself is difficult to articulate. 

Lynch describes resource discovery as a complex collection of activities ranging from locating a well-specified information to iterative research activities, that can involve the identification
of potentially relevant resources, organization and ranking of resources, and resource exploration~\cite{lynch1995networked}. Proper and Bruza apply the term ``information discovery'' in the context of  the identification and retrieval of relevant information from electronic sources~\cite{proper1999information}. 

In the field of cognitive psychology, Jerome S. Bruner~\cite{bruner1961act} defines information discovery as ``all forms of obtaining knowledge for oneself by the use of one's own mind.'' I build on Bruner's definition to underline the importance of the cognitive processes that govern information discovery. Therefore, I consider \textit{information discovery} as a process of obtaining knowledge from digital sources that can involve complex mental tasks and information behavior.  
\pagebreak

Information behavior refers to the totality of ways in which humans interact with information~\cite{wilson2000human}. It can enable and support information discovery when targeted at information maintenance and augmentation. This type of information behavior is also known as \textit{digital curation}.

Similar to the term ``information discovery'', the term ``digital curation" is perceived differently across disciplines and among researchers. In this thesis, I use the definition proposed by Giaretta~\cite{giaretta2006dcc} and adopted by the Digital Curation Centre\footnote{The Digital Curation Centre is a UK-based organization established to support expertise and practice in digital curation and preservation across communities of practice.} which states that digital curation is a process of maintaining and adding value to an existing body of information to improve its future use and retrieval.   

Information discovery can take on many forms. Web users might be hoping to find particular pieces of information, such as show times and phone numbers, to satisfy specific information needs~\cite{proper1999information}. Alternatively, they might be lacking well-articulated information needs, so they engage in opportunistic browsing~\cite{lindley2012s}. Sometimes people discover information online without even looking for it~\cite{bates1986exploratory}. The nature of information discovery can vary, and therefore, it requires elaborate tool support. The functionality required for information discovery and curation can also be distributed among multiple applications, which often leads to tools that provide integrated solutions. With people having such diverse information needs and methods of looking for information, designing for information discovery is a challenging task~\cite{conaway2010designing, marchionini2006exploratory}.

My research goal is to gain an understanding of how existing tools support digital information discovery and curation addressing the problem of designing Web applications for information discovery. While several researchers propose frameworks targeted at designing information discovery systems~\cite{proper1999information, kerne2004information}, the importance of information curation in the realm of information discovery has been largely overlooked despite the rapidly increasing popularity of socially-curated information spaces. Moreover, much of the existing work that focuses on how people look for and discover information online~\cite{bates1986exploratory, choo2000information, ellis1989behavioural, kellar2006goal, lindley2012s, morrison2001taxonomic, sellen2002knowledge} fails to examine concrete features of existing Web-based information discovery applications that empower real-world users. More research is necessary to determine how different tools and their features provide fundamental support for information discovery and curation.

To enhance information seeking and curating experiences and support users' interactions, I extend existing research by (1) deriving factors that enable information discovery and curation and relating them within a framework, (2) using the framework to establish a set of questions that can be used when evaluating and designing new applications, (3) iteratively evaluating the framework by using it to study and describe current Web applications as well as to design a new application, which in turn helped refine the framework of factors and questions, and (4) relating the framework to user information discovery and curation motives that drive the underlying usage of many Web-based applications. 

This thesis is organized as follows. My methodology and the process of building and refining a conceptual framework is documented in Chapter~\ref{chapter:methodology}. Chapter~\ref{chapter:chapter_related_work} highlights some of the studies and technologies related to information discovery and curation tasks. Chapter~\ref{chapter:old_framework} describes preliminary attempts at building the conceptual framework and outlines their shortcomings. Chapter~\ref{chapter:framework} outlines the conceptual framework and questions that enable digital information discovery and support curation, including specific examples from real-world Web applications. In Chapter~\ref{chapter:evaluation}, I illustrate the framework validation process, demonstrate how the framework can be used to reveal missing features in tools, and propose new directions for development with relation to user goals. I then showcase how the framework can be used for Web application design in Chapter~\ref{chapter:application}. Chapter~\ref{chapter:implications} summarizes the implications for research and practice. This is followed by future work and conclusions in Chapter~\ref{chapter:future_work}.




