%- THESIS.TEX
%-
%- An example file on how to use vthesis.cls
%- Vassilis Dimakopoulos, 1995 & 1996
%-
%- Use it as you would use report.sty (or .cls).
%-
%- The differences/additions are:
%   - the page layout is set according to Uvic Dec. 1994 rules
%-  - the construction of the `special' pages (titlepage, abstract,
%-       withhold form, etc) is automatic
%-  - the numbering (arabic & roman) of pages is also automatic
%-  - there are aesthetic changes (I call them improvements :-) to
%-       page headers, figure and table captions, etc.
%-  - a few simple entities have been defined
%     o  A \begin{proof} ... \end{proof} environment
%     o  Two simple lists: alphalist and numlist. numlist
%-       (\begin{numlist} \item .. \end{numlist}) is the same
%-       as `enumerate' but puts the numbers in boldface which
%-       looks good to me. Same goes for alphalist but instead of
%-       numbers you get (a), (b), etc.
%-
%- Please double check with the official thesis guidelines.
%- Report problems/questions to dimako@ece.uvic.ca
%-

\documentclass[12pt]{vthesis}   %- If no size given, 11pt is the default
                                %- Uvic now accepts as low as 10pt
\bibliographystyle{ieeetr}

\usepackage{graphics}           %- Or whatever else you need
\usepackage{graphicx}
\usepackage{epsfig}
\usepackage[dvips]{color}


%-----------------------------------------------------
%- Start the text already
%-----------------------------------------------------
\begin{document}

%-
%- Numbering is automatic: arabic & roman page numbers are put in the
%- appropriate pages. Roman paging starts at the first \chapter command.
%-

%-----------------------------------------------------
%- Stuff needed for title page and for other pages too
%-----------------------------------------------------

%\thesistype{Dissertation}       %- If not used, it defaults to "Thesis"
\thesistitle{THESIS TITLE}  %- Use capital letters
\name{CANDIDATE NAME} %- Candidate's name, uppercase is good
\degree{Master of Science}    %- What you're after

%-
%- If you are in a department other than ECE, use the command
%- \department{..}, e.g.
%-
\department{Department of Computer Science}


%- Here give previous degrees

%\prevdegree{M.A.Sc, University of Victoria, 1992}
\prevdegree{B.Sc., University of Victoria, 2002}

%- Let latex know about the committee members,
%-    in the order you want them

\committeemember{Dr.~******, Supervisor,\\ Dept.~of
Computer Science} \committeemember{Dr.~******, Member,\\
Dept.~of Computer Science} \committeemember{Dr.~******,
Member,\\ Dept.~of Computer Science}
\committeemember{Dr.~********, External Examiner}

%- Let latex know the supervisor's name

\supervisor{Dr.~Margaret-Anne Storey}

%-
%- NOW, make the title page
%-

\maketitlepage

%-----------------------------------------------------
%- Do your abstract as normal
%-----------------------------------------------------

\begin{abstract}
Something interesting

\end{abstract}

%--------------------------------------------------------------
%- Add table of contents, list of figures, tables etc as normal
%--------------------------------------------------------------

\tableofcontents
\listoffigures
\listoftables


%--------------------------------------------------------------
%- Here is how to add three (optional) pages
%- If you do not like what you get, do it manually
%--------------------------------------------------------------

%\begin{notation}
%\end{notation}

\begin{acknowledgement}
\end{acknowledgement}

\begin{dedication}
\end{dedication}

%--------------------------------
%- From now on you're on your own
%--------------------------------

%
\include{intro}        %- include your chapters

\include{FutureWork}

\bibliography{refs}

\appendix              %- maybe some appendices
%\include{app1}
%\include{app2}

%--------------------------------------------------------------
%- Here is a semi-automatic VITA page construction.
%- If you don't like the page layout you have to make your own.
%--------------------------------------------------------------

\vita                  %- This commands starts the vita page(s)

%- Now call \fullinfo with four arguments:
%-    lastname, givenname, birthplace, and birthdate.
%- I think that the birthdate is according to Dec. 1994 rules
%-    is optional.
%- So, I also provide the command:
%-    \halfinfo{lastname}{givenname}{birthplace}


\halfinfo{Mouse}{Mickey}{California}

%-
%- Next, the education:
%- \begin{education} ... \end{education}
%- Each entry must be entered as \entry{university}{years}}
%-

\begin{education}
\entry{University of Victoria}{\ldots to \ldots}
\end{education}

%-
%- Next the degrees:
%- \begin{degrees} ... \end{degrees}
%- Now \entry has 3 arguments: \entry{title}{university}{year}
%-

\begin{degrees}
\entry{B.Sc.}{ University of Somewhere}{\ldots}
\end{degrees}

%-
%- Next the awards:
%- \begin{awards} ... \end{awards}
%- Entries look like: \entry{award}{year}
%-

\begin{awards}
\entry{award}{date}
\end{awards}

%-
%- Now the publications.
%- I broke it 3 parts: journal, conference, and `submitted_to'.
%- Use as:   \begin{jpubs} \item ... \item ... \end{jpubs}
%-           \begin{cpubs} \item ... \item ... \end{cpubs}
%-           \begin{spubs} \item ... \item ... \end{spubs}
%- Example:
%-

%\begin{jpubs}
%\item V.V. Dimakopoulos, G. Sourtziotis, A. Paschalis and D. Nikolos,
%   ``On TSC Checkers for $m$-out-of-$n$ Codes'',
%   {\it IEEE Transactions on Computers\/},
%   Vol.\ 44, No.\ 8, pp.\ 1055--1059, Aug.\ 1995.
%\item Another one, ...
%\end{jpubs}

%\begin{cpubs}
%\item V.V. Dimakopoulos and N.J. Dimopoulos, ``Optimal Total Exchange
%   in Cayley Graphs'', {\itshape submitted to\/} SIAM Journal on
%   Computing.
%\end{cpubs}

%\begin{spubs}
%\item V.V. Dimakopoulos and N.J. Dimopoulos, ``Optimal Total Exchange
%   in Cayley Graphs'', {\itshape submitted to\/} SIAM Journal on
%   Computing.
%\end{spubs}


%--------------------------------------------------------------
%- We can also produce the copyright and the withhold page.
%- These pages are unnumbered and should be printed at the
%- end. To construct the withhold page we need to know when
%- the successful defence was held. Also need the date for
%- the copyright page.
%--------------------------------------------------------------

\makecopyrightpage{30 Dec 2002}   %- Date below signature
%\makewithholdpage{30 Jan 2002}  %- When we defended

\end{document}
